% Options for packages loaded elsewhere
\PassOptionsToPackage{unicode}{hyperref}
\PassOptionsToPackage{hyphens}{url}
\documentclass[
]{article}
\usepackage{xcolor}
\usepackage{amsmath,amssymb}
\setcounter{secnumdepth}{-\maxdimen} % remove section numbering
\usepackage{iftex}
\ifPDFTeX
  \usepackage[T1]{fontenc}
  \usepackage[utf8]{inputenc}
  \usepackage{textcomp} % provide euro and other symbols
\else % if luatex or xetex
  \usepackage{unicode-math} % this also loads fontspec
  \defaultfontfeatures{Scale=MatchLowercase}
  \defaultfontfeatures[\rmfamily]{Ligatures=TeX,Scale=1}
\fi
\usepackage{lmodern}
\ifPDFTeX\else
  % xetex/luatex font selection
\fi
% Use upquote if available, for straight quotes in verbatim environments
\IfFileExists{upquote.sty}{\usepackage{upquote}}{}
\IfFileExists{microtype.sty}{% use microtype if available
  \usepackage[]{microtype}
  \UseMicrotypeSet[protrusion]{basicmath} % disable protrusion for tt fonts
}{}
\makeatletter
\@ifundefined{KOMAClassName}{% if non-KOMA class
  \IfFileExists{parskip.sty}{%
    \usepackage{parskip}
  }{% else
    \setlength{\parindent}{0pt}
    \setlength{\parskip}{6pt plus 2pt minus 1pt}}
}{% if KOMA class
  \KOMAoptions{parskip=half}}
\makeatother
\usepackage{longtable,booktabs,array}
\usepackage{calc} % for calculating minipage widths
% Correct order of tables after \paragraph or \subparagraph
\usepackage{etoolbox}
\makeatletter
\patchcmd\longtable{\par}{\if@noskipsec\mbox{}\fi\par}{}{}
\makeatother
% Allow footnotes in longtable head/foot
\IfFileExists{footnotehyper.sty}{\usepackage{footnotehyper}}{\usepackage{footnote}}
\makesavenoteenv{longtable}
\usepackage{graphicx}
\makeatletter
\newsavebox\pandoc@box
\newcommand*\pandocbounded[1]{% scales image to fit in text height/width
  \sbox\pandoc@box{#1}%
  \Gscale@div\@tempa{\textheight}{\dimexpr\ht\pandoc@box+\dp\pandoc@box\relax}%
  \Gscale@div\@tempb{\linewidth}{\wd\pandoc@box}%
  \ifdim\@tempb\p@<\@tempa\p@\let\@tempa\@tempb\fi% select the smaller of both
  \ifdim\@tempa\p@<\p@\scalebox{\@tempa}{\usebox\pandoc@box}%
  \else\usebox{\pandoc@box}%
  \fi%
}
% Set default figure placement to htbp
\def\fps@figure{htbp}
\makeatother
\setlength{\emergencystretch}{3em} % prevent overfull lines
\providecommand{\tightlist}{%
  \setlength{\itemsep}{0pt}\setlength{\parskip}{0pt}}
\usepackage{bookmark}
\IfFileExists{xurl.sty}{\usepackage{xurl}}{} % add URL line breaks if available
\urlstyle{same}
\hypersetup{
  hidelinks,
  pdfcreator={LaTeX via pandoc}}

\author{}
\date{}

\begin{document}

\subsection{Exam 7 High-Level Summaries 2025
Sitting}\label{exam-7-high-level-summaries-2025-sitting}

Rising Fellow

\begin{center}\rule{0.5\linewidth}{0.5pt}\end{center}

Copyright C 2024 by Rising Fellow LLC

All rights reserved. No part of this publication may be reproduced,
distributed, or transmitted in any form or by any means, including
photocopying, recording, or other electronic or mechanical methods,
withou the prior written permission of the publisher, except in the case
of brief quotations embodied in critical reviews and certain other
noncommercial uses permitted by copyright law. For permission requests,
write to the publisher at the address below.

Published By:

Rising Fellow United States,TX, 78006 www.RisingFellow.com

Published in the United State

\begin{center}\rule{0.5\linewidth}{0.5pt}\end{center}

\(\begin{aligned}
&\frac{\text{Table of Contents}}{} \\
&\text{Table of Contents} \\
&\text{Mack-Benktander} \\
&\text{Hürlimann} \\
&\text{Brosius} \\
&\text{Friedlanc} \\
&\text{Clark} \text{20}  \\
&\mathrm{Mack-Chain~Ladder} \\
&\text{Venter Factors} \text{29}  \\
&\text{Shapland} \\
&\text{Siewert} \\
&\text{Sahasrabuddhe} \text{48}  \\
&\text{Teng and Perkins} \\
&\text{Meyers} \\
&\text{Taylor} \\
&\text{Verrall} \\
&\text{Marshall} \\
&\frac1{\otimes\text{Rising Fellow}}\text{High-Level Summaries|i}
\end{aligned}\)

\begin{center}\rule{0.5\linewidth}{0.5pt}\end{center}

\begin{center}\rule{0.5\linewidth}{0.5pt}\end{center}

\subsection{Credible Claims Reserves: The Benktander
Method}\label{credible-claims-reserves-the-benktander-method}

\subsection{Overview}\label{overview}

The Mack - Benktander paper is a calculation-heavy paper. Most
importantly, you need to be able to estimate Benktander reserves a
number of different ways based on how the problem is written.Another key
concept to remember is that the Benktander ultimate loss estimate is a
credibility-weighting of the chain ladder and expected loss ultimates.

\subsubsection{Benktander Method}\label{benktander-method}

The Benktander method can be calculated as a second iteration of the BF
procedure or as a credibility weighting of the Chain Ladder and Expected
Loss ultimates (see the``Mack-Benktander -Benktander Method'' recipe).d

\subsection{Benktander as a second iteration of the BF
procedure.}\label{benktander-as-a-second-iteration-of-the-bf-procedure.}

Iteration 1 - Bornhuetter-Ferguson

\[Ult_{_{BF}}=Loss+(1-\%Paid)\times Prem\times ELR\]

Iteration 2-Benktander \[\boxed{U_{_{GB}}=C_{_k}+q_{_k}U_{_{BF}}}\]

\[U_{_{GB}}=Loss+(1-\%Paid)\times Ult_{_{BF}}\]

Benktander as a credibility-weighting of the Chain Ladder and Expected
Loss Ultimates.

Chain Ladder Ultimate:

\[Ult_{cL}=Loss\times CDF\]

C Uca Pk

Benktander:

\(q_{k}^{*}=1-\frac{1}{CDF}\)

\(U_{GB}=\left(1-q_{k}^{*}\right)^{2}U_{CL}+q_{k}^{*2}U_{0}\)

Benktander as a credibility-weighting of the Chain Ladder and BF
Reserves

\(R_{GE} = (1 - q_s) R_{CE} + q_s R_F\)

\subsubsection{Advantages of the Benktander
Method}\label{advantages-of-the-benktander-method}

.Outperforms the BF and Chain Ladder methods in many circumstances ·The
MSE of the Benktander reserve is almost as small as that of the optimal
credibility reserve

\begin{center}\rule{0.5\linewidth}{0.5pt}\end{center}

\subsubsection{Iterated BF Method}\label{iterated-bf-method}

The Benktander method is a second iteration ofthe BF procedure.Thisis
how the iteration works

1.Start with an ultimate loss estimate, \(U^{(\mathrm{m})}\) .For
\(U^{(0)}\) , use the expected loss estimate. 2.Apply theBFprocedure to
get a new loss reserve estimate:

\pandocbounded{\includegraphics[keepaspectratio]{https://storage.simpletex.cn/view/fImhLRSAUr226YAVU8tYM0RRvHBgd3VzW}}

3.Get a new ultimate loss estimate by adding thelosses-to-date to the
reserve.This is the starting ultimate for the next iteration:

\pandocbounded{\includegraphics[keepaspectratio]{https://storage.simpletex.cn/view/fmIlLBvguAnQazhhS0O74LsCuii0si4Pm}}

The ultimate loss estimate \((\mathbf{U}^{(\mathrm{m})})\) can be
rearranged as a credibility weighting of the Chain Ladder ultimate
\((\mathrm{U}_{\mathrm{CL}})\) and expected loss ultimate
\((\mathrm{U}_{0})\) .Also, the loss reserve estimate
\((\mathbf{R}^{(\mathrm{m})})\) can be rearranged as a credibility
weighting of the ChainLadder reserve \((\mathbf{R}_{\mathrm{CL}}\) ) and
the BF reserve \(\mathrm{R}_{\mathrm{BF},}\)

\pandocbounded{\includegraphics[keepaspectratio]{https://storage.simpletex.cn/view/fisnayqF3FgBXFaGCzBeLfsSHd4Zodyho}}

As the number of iterations increases,the weight on the chain ladder
method increases until it converges to the chain ladder method entirely
(as \(m\rightarrow\infty\) ).

\subsection{Recipes for Calculation
Problems}\label{recipes-for-calculation-problems}

Benktander Method

\begin{center}\rule{0.5\linewidth}{0.5pt}\end{center}

\subsection{Credible Loss Ratio Claims
Reserves}\label{credible-loss-ratio-claims-reserves}

\section{Overview}\label{overview-1}

The reserve estimate method in Hurlimannis a credibility-weighted method
that's very similar to the Mack (2000) method. The key difference is
that Huirlimann uses expected incremental loss ratios \((m_k)\) to
specify the payment pattern instead of using LDFs calculated directly
from thelosses.

Hurlimann uses two new reserving methods based on the loss ratio payout
factors,p

Individual Loss Ratio Reserve \((R^{ind})-\) Similar to the chain ladder
method

·Collective Loss Ratio Reserve \(R^{\text{cod}\prime\prime}\) Similar to
the Bornhuetter-Ferguson (BF) method

The key idea from Hurlimann is that \(R^{ind}\) and
\(R^{\mathrm{coll}}\) represent extremes of credibility on the actual
loss experience and we can calculate a credibility-weighted estimate
that minimizes theMSE of the reserve estimate.

\subsection{Credible Loss Ratio Claims
Reserve}\label{credible-loss-ratio-claims-reserve}

Individual Loss Ratio Claims Reserve (Rn

\[R^{ind}=\frac{\%Unpaid_{AY}\cdot Loss_{AY}}{\%Paid_{AY}}\]

\pandocbounded{\includegraphics[keepaspectratio]{https://storage.simpletex.cn/view/fLczcXNhgoagVNCKImWsc18LE8TFfVk6F}}

\(R^{ind}-100\%\) credibility on losses-to-date

\subsubsection{Collective Loss Ratio Claims Reserve
R}\label{collective-loss-ratio-claims-reserve-r}

\[\boxed{R^{call}=q_{i}\cdot Prem\cdot ELR}\quad R^{call}=96Unpaid_{AY}\cdot Premium_{AY}\cdot ELR\]

\(R^{\mathrm{cod}\prime}-0\%\) credibility on losses-to-date

\subsubsection{Credibility-Weiqhted Reserve
Estimate}\label{credibility-weiqhted-reserve-estimate}

We can calculate a new, credibility-weighted estimate,based on
\(R^{ind}\) and \(R^{\mathrm{rob}}\) that minimizes the mean squared
error (MSE) and variance of the loss reserve estimate.

\[\boxed{R_i=Z_i\cdot R_i^{ind}+(1-Z_i)\cdot R_i^{coll}}\]

Hurlimann uses three different credibility methods:

\pandocbounded{\includegraphics[keepaspectratio]{https://storage.simpletex.cn/view/fLZZG2tq9M61hxgT5frk65pmTRLzVEEdP}}

Benktander (also in Mack (2000)

:Neuhaus

Optimal credibility weighting (minimizes MSE)

\begin{center}\rule{0.5\linewidth}{0.5pt}\end{center}

Straightforward calculation of the optimal credibility weight ·Different
actuaries get the same result using the collective loss ratio claims
reserve with the same premiums (BF method requires an ELR assumption)

\subsubsection{Advantage of the optimal credibility weiqht reserve
(Rp}\label{advantage-of-the-optimal-credibility-weiqht-reserve-rp}

·Minimizes MSE andvariance of theloss reserve estimate ONote: the MSE
from Benktander and Neuhaus are close to the optimal credibilityMSE

\subsection{Optimal Credibility
Weights}\label{optimal-credibility-weights}

Hurlimann derives the optimal credibility weights in sections 4-6.Many
of the formulas are intermediary formulas in the derivation.For the
exam,I would focus primarily on the final, simplified optimal
credibility weight as well as the generalized optimal credibility
formula (see the Optimal Credibility Weights'' recipe).

If we assume that the variance of the ultimate loss is the same as the
variance of the burning cost ultimate loss estimate,
\(\operatorname{Var}(U_{i})=\operatorname{Var}(U_{i}^{BC})\) , we get
the simplified optimal credibility weight formula. If we make a
different assumption, then you need to use the generalized version of
the formula

You should definitely know the simplified optimal credibility weight
formula.A potential twist to a question would be to use a different
assumption, such as
\(\operatorname{Var}(U_{i})=2\times\operatorname{Var}(U_{i}^{BC})\)
,and then use the generalized formula to calculate the optimal
credibility weight.

\subsection{Application to Standard
Approaches}\label{application-to-standard-approaches}

Hiurlimann derived the optimal credibility weight formula for the loss
ratio claims reserve approach. It can also be used with a more
traditional approach, using LDFs to calculate the payout pattern
\((p_{i}^{CL})\) .With the LDF-based payout pattern, you can then
calculate the reserve estimate as a credibility-weighting of the Chain
Ladder and Cape Cod (or BF) reserve estimates using the Benktander,
Neuhaus or optimal credibility weights.

Credibility-Weighted Cape Cod Approach

·Use LDFs to calculate the payout pattern \((p_{i}^{CL})\) ·Calculate
the ELR using the Cape Cod method

Credibility-Weiqhted Bornhuetter Ferquson Approach

·Use LDFs to calculate the payout pattern \((p_{i}^{CL})\) ·The ELR is
an assumption

\subsection{Recipes for Calculation
Problems}\label{recipes-for-calculation-problems-1}

·Credible Loss Ratio Claims Reserve Optimal Credibility Weights

\begin{center}\rule{0.5\linewidth}{0.5pt}\end{center}

\subsection{Loss Development Using
Credibility}\label{loss-development-using-credibility}

\subsection{Overview}\label{overview-2}

Brosius introduces the Least Squares method for estimating loss reserves
and compares this method to the traditional Chain Ladder and the
Budgeted Loss (Expected Loss) methods. The key theme of this paper is
that the Least Squares method is a credibility weighting of the Link
Ratio (Chain Ladder) and Budgeted Loss methods.

\subsubsection{Least Squares Method}\label{least-squares-method}

The Least Squares method fits a regression line through the data to
estimate developed losses ( \(\hat{y}\) )

\pandocbounded{\includegraphics[keepaspectratio]{https://storage.simpletex.cn/view/fh6k4wTwSwb5S8zDnWpbbfKO4q1SgMKSk}}

See the Brosius Least Squares'' recipe.

\subsubsection{Comparison of Methods}\label{comparison-of-methods}

\pandocbounded{\includegraphics[keepaspectratio]{https://storage.simpletex.cn/view/fUinwOpMC3U10kCFghRA4DTENQQclPKGS}}

\subsection{Least Squares as Credibility
Weighting}\label{least-squares-as-credibility-weighting}

The Least Squares method is a credibility weighting of the Link Ratio
and Budgeted Loss methods. The Link Ratio and Budgeted Loss methods
represent the extremes:

Link Ratio: Places \(100\%\) credibility on loss experience and \(0\%\)
on expected losses ·Budgeted Loss: Places 096 credibility on loss
experience and \(100\%\) on expected losses

\subsection{Least Squares Credibility
Formula}\label{least-squares-credibility-formula}

The Least Squares method is flexible and places more (or less)
credibility on the loss experience as appropriate.

\begin{center}\rule{0.5\linewidth}{0.5pt}\end{center}

Below are the key formulas for calculating the credibility on the link
ratio method.The factor \(\mathbf{C}\) is just the LDF for the link
ratiomethod.The credibility is the ratio of \(b\) to the LDF.The closer
\(b\) is to the LDF, the higher the credibility weighting the least
squares method places on the link ratio method

CLDP-\/---

Credibility-weighted formula

j=Zx+(1-Z)xE{[}y{]}
\[\hat{y}=Z\times LDF\times x+(1-Z)\times\mathrm{E}\Big[\:y\:\Big]\]

\subsubsection{Special Cases}\label{special-cases}

If the regression line fits through the origin,then \(\mathbf{a}=0\)
,resulting in the Chain Ladder method where \(\hat{y}=bx\)

If \(X\) and y are completely uncorrelated, then \(b=0\) ,resulting in
the Budgeted Loss method where \(\hat{y}=a\) . The Bornhuetter-Ferguson
method is a special case of Least Squares where \(b=1\) .The BF method
can be problematic if negative loss development is expected. The Least
Squares method would allow b to adapt to the observed data.

Potential Problems (and Fixes) with Least Squares.

The intercept is negative \((a<0\),

This causes the estimate of developed losses ( \(\hat{y}\) ) to be
negative for small values of \(X.\) Solution: Use the link ratio method
instead

\subsection{\texorpdfstring{The slope is negative
\((b<0)\)}{The slope is negative (b\textless0)}}\label{the-slope-is-negative-b0}

This causes the estimate of \(y\) to decrease as \(X\) increases
·Solution:Use the budgeted loss method instead

\subsubsection{Key Assumptions for Least
Squares}\label{key-assumptions-for-least-squares}

Least Squares assumes a steady distribution of random variables X and Y

·Least Squares is inappropriate if there's a systematic shift in the
book of business

\subsubsection{Advantages of Least
Squares}\label{advantages-of-least-squares}

·Least Squares is more flexible than the link ratio, budgeted loss, and
BF methods. ·Least Squares is a credibility weighting of the link ratio
and budgeted loss estimates. It gives more (or less)credibility to the
loss experience \((x)\) as appropriate.

\begin{center}\rule{0.5\linewidth}{0.5pt}\end{center}

Least Squares produces more reasonable results when the data has severe
random, year-to-year fluctuations (e.g.~a small book of business or thin
data)

\subsubsection{Adjustments to the data when using Least
Squares}\label{adjustments-to-the-data-when-using-least-squares}

·When using incurred loss data, the data should be adjusted for
inflation so that all accident years are on a constant-dollarbasis.

If there is significant growth in the book of business, you should
divide the data by an exposure basis.

\section{Hugh White's Question}\label{hugh-whites-question}

If reported losses \((x)\) come in higher than expected, the different
methods will estimate different changes to the outstanding loss reserve:

·Budgeted LossMethod (fixed prior case)-Theultimate loss estimate is
fixed,so we decrease the loss reserve estimateby the same amount as the
unexpected increase in reported losses. This method treats the increased
loss as losses coming in faster than expected

BF Method-The ultimate loss estimate increases by the amount losses were
greater than expected The Ioss reserve is unchanged. The BF method
treats the unexpected increased loss as a random fluctuation (e.g.~a
large loss).

Link Ratio Method (fixed reporting case) - The ultimate loss estimate
increases in proportion to the excess losses by applying the LDF, so we
increase the loss reserve estimate. This method assumes that a fixed
percentage of ultimate losses is reported, so if reported losses
increases, the ultimate loss estimate will increase proportionally.

\subsection{Theoretical Models - Testing Least
Squares}\label{theoretical-models---testing-least-squares}

The purpose of this section is to test the least squares model against a
few different theoretical loss models. With a theoretical model,we can
use Bayes'Theorem to calculate the``correct''loss model and then see
whether theleast squares,budgeted loss orlink ratio modelshave the same
form as the Bayesian approach

\begin{longtable}[]{@{}lll@{}}
\toprule\noalign{}
Model & Form & Model Constraints \\
\midrule\noalign{}
\endhead
\bottomrule\noalign{}
\endlastfoot
Least Squares & \(\\hat{y}=a+bx\) & \(a=0\) \\
Link Ratio & \(\\hat{y}=ax\) & \(b=0\) \\
Budgeted Loss & \(\\hat{y}=a\) & \\
Bornhutter-Ferguson & \(\\hat{y}=a+x\) & \(b=1\) \\
\end{longtable}

\subsubsection{Simple Model}\label{simple-model}

The number of ultimate claims incurred (Y) is either O or 1 with equal
probability

·If there is a claim (Y=1) ,there is a \(50\%\) chance it's reported by
year end (X)

Using BayesTheorem,the best estimate of ultimate claims given \(x\) is
\(\hat{y}=\frac{1}{3}+\frac{2}{3}x\) .This is the form \(\hat{y}=a+bx\)
, so only the Least Squares method is compatible

\begin{center}\rule{0.5\linewidth}{0.5pt}\end{center}

Poisson -Binomial Model

·The number of ultimate claims incurred (Y) is Poisson with mean \(\mu\)
·Any given claim has probability \(d\) of being reported by year end

Using Bayes'Theorem, the best estimate of ultimate claims is
\(\hat{y} = x+ \mu ( 1- d)\) . This is the same form as both the Least
Squares method and BF method, since \(b=1\)

\subsubsection{Negative Binomial -Binomial
Model}\label{negative-binomial--binomial-model}

·The number of ultimate claims incurred (Y) is Negative Binomial with
parameters \((r,p)\)

·Any given claimhas probability \(d\) of beingreported by year end

This model also has a Baysian estimate with the same form as the Least
Squares method, but the other methods will be incorrect

Linear Approximation (Bayesian Credibility Approach)

We can only calculate the true Bayesian estimate by assuming a
distribution for Y and \(X|Y\) but that's not practical. Instead, we're
going to find the best linear approximation to the Bayesian estimate of
ultimate losses with Bayesian credibility, \(L(x)\)

\[L(x)=\begin{pmatrix}x-\operatorname{E}[X]\end{pmatrix}\frac{\operatorname{Cov}(X,Y)}{\operatorname{Var}(X)}+\operatorname{E}[Y]\]

Below is how a large reported loss (increasing x) can change the loss
reserves, corresponding with the three different answers to Hugh White's
question. For \(x>\)E\([X]\)

\(\operatorname{Cov}(X,Y)<\operatorname{Var}(X)\) loss reserve decreases
· \(\operatorname{Cov}(X,Y)=\operatorname{Var}(X)\) loss reserve
unaffected (ultimate loss increases by the increase to x) ·
\(\operatorname{Cov}(X,Y)>\operatorname{Var}(X)\) loss reserve increases

Using loss data, we can estimate \(\operatorname{Cov}(X,Y)\) ,
\(Var(X)\) , and E{[}Y{]}, which gets us right back to the Least Squares
method

\[L(x)=\hat{y}=(x-\overline{x})\frac{\overline{xy}-\overline{x}\cdot\overline{y}}{\overline{x^{2}}-\overline{x}^{2}}+\overline{y}\]

The key point is that the least squares method is thebest linear
approximation to theBayesian estimate, although there will be sampling
error in the parameter estimates of \(a\) and \(b.\)

Using simulated data from the Poisson-Binomial model, the Least Squares
method fits the data better than thelinkratiomethod and has a lowerMSE

\begin{center}\rule{0.5\linewidth}{0.5pt}\end{center}

\section{Bayesian Credibility}\label{bayesian-credibility}

If the book of business changes significantly, we can't use the regular
Least Squares method. But, if we make a few assumptions about the
expected ultimate losses (Y) and the percent reported \((\frac{X}{Y})\)
, then we can calculate a Bayesian credibility estimate of ultimate
losses (See the Brosius Bayesian Credibility'' recipe)

\subsection{Caseload Effect}\label{caseload-effect}

The regular Bayesian credibility formula assumes that the expected
percent of losses reported is the same no matter how large ultimate loss
(Y) is. The caseload effect says that if ultimate loss is higher, then
we would expect a lower percent of losses to be reported at time x (See
the ``Brosius - Caseload Effect'' recipe).

Bayesian credibility still works,but the Bayesian credibility formula
needs to be modified. Instead of using a fixed percent reported, the
expected percent reported is lower if ultimate losses (Y) are
higher.Below is a graphical view of the caseload effect and how the
caseload effect estimate compares to the unmodified chain ladder
estimate.

·I \(\mathbf{f}_{\mathbf{x}}>\mathbf{E}[\mathbf{X}]\) ,the caseload
estimate will be higherthan the unmodified chain ladder estimate

· If x\textless{} E{[} X{]} ,the caseload estimate willbelower than the
unmodified chain ladder estimate.

\pandocbounded{\includegraphics[keepaspectratio]{https://storage.simpletex.cn/view/fSyoXRuRQtPeWhAu9CaUsNdKo90Yq6RES}}

\subsection{Recipes for Calculation
Problems}\label{recipes-for-calculation-problems-2}

·Least Squares Method Bayesian Credibility .Caseload Effect

\begin{center}\rule{0.5\linewidth}{0.5pt}\end{center}

\subsection{Reserving for Reinsurance}\label{reserving-for-reinsurance}

\subsection{Overview}\label{overview-3}

Below are some of the key topics to understand from the Friedland paper:

·The types and functions of reinsurance ·Differences in data and
reserving between reinsurers and primary insurers ·The comparison of the
volatility in development factors and patterns between reinsurance,
primary insurance, and more ·How different reinsurance contracts
interact and how to calculate ceded loss reserves

Appropriate reserving for reinsurance is important for the following
reinsurer stakeholders:.

Internal Management - Sound reserves affect all areas of reinsurer
operations (pricing, underwriting, strategic planning, financial
decision-making, \ldots). Investors -Appropriately stated reserves are
essential so that investors can properly evaluate the reinsurer's
balance sheets and income statements for their decision-making Insurance
Regulators -Regulators rely on the financial statements of reinsurers to
properly supervise the reinsurance market. Rating Agencies -If a
reinsurer reports significant adverse reserve developments over time
that reduce capital leaving the reinsurer in a weakened position, it
could face a rating downgrade.

\subsection{Functions of Reinsurance}\label{functions-of-reinsurance}

·Promote Stability -Helps a ceding company stabilize loss experience
over time and protect the ceding company from large unforeseeable
losses. This can decrease the probability of ruin. Increase Capacity -
By ceding a portion of all policies or its larger policies, a ceding
company can increase its capacity to write more business,particularly at
higher policy limits Protect against Catastrophes Reinsurance can
protect ceding companies from catastrophic loss events as well as
protect against casualty loss occurrences with multiple insureds (like
terrorism) ·Manage Capital and Solvency MarginoReinsurance can help a
ceding company pass the risk of large losses to the reinsurer which
frees up capital since less capital will be required to support the
policies written. OThe ceded commission acts as a transfer of statutory
surplus from the reinsurer to the cedent, which can manage financial
results OPremium ceded also reduces the cedent's net premium-to-surplus
ratio (solvency margin) which allows the cedent to write more business.

\begin{center}\rule{0.5\linewidth}{0.5pt}\end{center}

Access Technical Expertise Reinsurers have technical expertise in
underwriting, marketing claims, loss prevention, pricing, and entering
new lines/regions that can help small insurers.

Other Functions - Reinsurance can help a ceding company withdraw from a
line of business, geographic area, or production source

\subsection{Types of Reinsurance}\label{types-of-reinsurance}

Reinsurance is categorized as Treaty or Facultative and Proportional or
Non-Proportional

\subsection{Treaty Reinsurance}\label{treaty-reinsurance}

The ceding company cedes all business arising from the lines of business
that fall within the terms of the treaty subject to treaty limits.
·There is no underwriting by the reinsurer of individual risks within
the treaty terms The cedent has an obligation to cede a risk under the
treaty terms and the reinsurer has an obligation to automatically accept
it.

\subsubsection{Facultative Reinsurance}\label{facultative-reinsurance}

Both the ceding company and reinsurer have the option (faculty) to
accept or reject individual submissions and an individual reinsurance
agreement is negotiated for each policy ceded

·Primarily used to increase capacity, typically for high-value and
hazardous commercial risks.

\subsubsection{Proportional Reinsurance}\label{proportional-reinsurance}

·Increases capacity and manages capital and solvency margins to provide
surplus relief

Both premiums and losses are shared between the cedent and reinsurer
based on the cession percentage. The reinsurer pays a ceding commission
to reimburse the cedent for expenses underlying the policy

\subsection{Types of Proportional
Reinsurance}\label{types-of-proportional-reinsurance}

·Quota Share -The ceding company cedes an agreed percentage of each risk
(premiums and losses) to the reinsurer for the lines of business in the
contract, typically a treaty

Surplus Share - The cedent cedes the surplus amount of a risk above the
retained line subject to a maximum ceded percentage and limit. The
proportion ceded is different for different insured policies according
to the underlying policy limits and acts like a variable quota share.

Policy Limit - Retained Line Proportion Ceded = Policy Limit

\subsubsection{Non-Proportional Reinsurance (Excess of
Loss)}\label{non-proportional-reinsurance-excess-of-loss}

·Provides stability by protecting losses above a limit for risks ceded
·Loss ceded is based on the size of loss and the premium is not
proportional to the coverage limits

\begin{center}\rule{0.5\linewidth}{0.5pt}\end{center}

·Excess per Risk -Excess-of-loss reinsurance above a retention for each
risk 0Primarily to protect property exposures (e.g.~7M excess 3M on
property policies with 10M policy limits) OAllows ceding companies to
write larger risks (increase capacity)

Excess per Occurrence and Catastrophe -For a loss occurrence,the ceding
company retains the retention and cedes the loss excess the retention to
the reinsurer up to the reinsurance limit.

OExcess per Occurrence -Protects a cedent from the accumulation of
losses in a single loss occurrence. OCatastrophe Reinsurance -Form of
Excess per Occurrence for a single catastrophio event or series of
events.Most allow for reinstatement after afull limit loss.

Annual Aggregate Excess of Loss (stop-loss)- Guarantees a ceding
company's losses won't exceed a predetermined threshold (percent of
premium or fixed dollar amount)

OThe reinsurer indemnifies for losses above the aggregate value
OProtects net results (other reinsurance inures to the benefit of the
Agg. Excess of Loss) OThe best option to protect capital but can be very
expensive or unavailable

Clash - Casualty reinsurance contract that attaches above all other
policy limits.It protects a ceding company when there are multiple
claims from multiple insureds for the same catastrophe and its
reinsurance policy doesn't fully reimburse the insurer.

oComponents of a clash event Loss must have multiple policies by one
insured or similar policies held by multiple insureds The losses are
traceable to and the direct consequence of a specific event ·The event
must take place within a specific timeframe

\subsection{Finite Risk Reinsurancee}\label{finite-risk-reinsurancee}

This type of reinsurance takes the time value of money into account.
Features include

·Risk transfer and risk financing in a multi-year contract ·Incorporates
the time value of money and investment income ·Limited assumption of
risk by the reinsurer ·Reinsurer and ceding company share results

Includes run-off solutions, which transfer reserve development risk to
the reinsurer. Reasons for run-off include corporate restructuring,
mergers 8 acquisitions, discontinuation of lines of business, erratic
changes in the valuation/cost of a liability,.

\subsubsection{Loss Portfolio Transfers}\label{loss-portfolio-transfers}

·Transfers all (or a portion) of liability for future loss payments on
losses already incurred

\begin{center}\rule{0.5\linewidth}{0.5pt}\end{center}

Relieves cedent of uncertainty in loss reserves and relieves capital

\subsubsection{Adverse Development
Cover}\label{adverse-development-cover}

Ceding company is reimbursed for losses excess a retention,but there's
no transfer of the loss reserves to the reinsurer Often used for Mergers
\& Acquisitions to transfer the risks of timing and adverse reserve
development

\subsection{Reinsurance Concepts and Contract
Provisions}\label{reinsurance-concepts-and-contract-provisions}

\subsection{Inure to theBenefit of}\label{inure-to-thebenefit-of}

The concept of ``inuring to the benefit of'' specifies whether a treaty
takes effect to the benefit of the reinsurer or the reinsured

For a reinsurance Treaty A:

●If other reinsurances apply first to reduce the loss subject to Treaty
A, then the other reinsurance contracts inure to the benefit of the
reinsurer of Treaty A. If other reinsurances are ignored with respect to
Treaty A (they don't lower the loss subject to Treaty A),then the other
reinsurance contracts inure to the benefit of the ceding company.

\subsubsection{Losses-Occurrinq Durinq vs
Risks-Attaching}\label{losses-occurrinq-durinq-vs-risks-attaching}

·Losses-Occurring During - Reinsurance coverage for all losses that
occur between the inception and expiration of the reinsurance contract,
regardless of when the underlying policy was issued. ·Risks-Attaching -
Reinsurance coverage for losses on underlying policies with inception
dates during the reinsurance contract's effective period. The underlying
policies ``attach'' to the contract.

\subsection{Subscription Percentage}\label{subscription-percentage}

A subscription policy is a reinsurance policy where risk is shared by
multiple reinsurers. Each reinsurer has a subscription percentage to the
contract.

Reasons for this approach include:

·When coverage is more than one reinsurer is willing to assume ·Allows
the cedent to diversify credit risk(the risk that a reinsurer can't pay
reinsurance recoveries)

\subsection{Commutation Clause}\label{commutation-clause}

A commutation is the cancellation of a reinsurance contract. The
reinsurer pays the present value of reinsurance recoveries not yet due
for the termination of the contract and all future obligations.

Both reinsurers and ceding companies have reasons for commutations

\begin{center}\rule{0.5\linewidth}{0.5pt}\end{center}

\subsubsection{For a Reinsurer}\label{for-a-reinsurer}

\subsection{For a CedingCompany}\label{for-a-cedingcompany}

To exit a line of business or region To manage reserves for transfer or
sale To avoid credit risk of a reinsurer To better manage claims and
claims-related expenses

To terminate a relationshipwith a ceding company To protect itself from
the insolvency of a ceding company To avoid disputes with the cedent
about future loss development

\subsubsection{Sufficient and Reliable
Data}\label{sufficient-and-reliable-data}

\section{Sufficiency}\label{sufficiency}

Data are sufficient if they included all information needed for the
actuarial work.

For the development method:

Reinsurance data may not be appropriate for the underlying development
method assumptions because of its manuscript nature (custom-written) and
due to operational changes at ceding companies,the reinsurer or both
Changes at the ceding company level can violate the assumption of
consistency in mix of business, attachment points, limits, claims
processing, etc.

\subsection{Reliability}\label{reliability}

Data are reliable if they are sufficiently complete, consistent, and
accurate for the purposes of the work

Data should be validated: Reviewed for consistency, completeness, and
accuracy.

Reliability challenges for reinsurers vs.~primary insurers:

Each ceding company/broker has different IT systems, terminology, etc
:Bordereau reporting can differ (types of data, labeling, frequency)
Reporting lags OClaims are reported first to ceding company before being
reported to the reinsurer oLong-tailed nature of certain reinsurance
like excess per risk and catastrophe reinsurance OBordereau reporting:
losses are only reported quarterly or more infrequent schedule Gaps in
reporting of critical claims and claims expense information by the
ceding company :Manuscript nature of reinsurance policies

\subsection{Homogeneity and Credibility of
Data}\label{homogeneity-and-credibility-of-data}

Homogeneity

Homogeneous Risk Group (HRG) - Set of (re)insurance obligations with
similar risk characteristics to allow for reliable valuation of unpaid
losses or technical provisions.

\begin{center}\rule{0.5\linewidth}{0.5pt}\end{center}

Data should be segmented into groups with similar characteristics of
loss experience such as consistency of coverage, similar
reporting/payment patterns, ability to develop appropriate case
outstanding for claims, severity, and volume of losses in the group.

\subsubsection{Credibility}\label{credibility}

Credibility-Ameasure of the predictivevaluefor a set of data.

Credibility for a homogeneous risk group increases with:

·Increasing homogeneity of the data within a group

·Increasing the volume of data in a group

If groups are too granularly defined, the volume of data in each group
may be too low for a reliable analysis.

Important Variables for Partitioning the Reinsurance Portfolio

Line of business (property, casualty \ldots) Type of contract
(facultative, treaty, \ldots) .Type of reinsurance cover (quota share,
excess per-occurrence, CAT, \ldots) ·Primary line ofbusiness -for
casualty Attachment point - for casualty Contract terms (flat-rated,
retro-rated, claims-made,\ldots{} ·Type of cedent (small, large,..)
:Intermediary

\section{Organization of Data by Experience
Period}\label{organization-of-data-by-experience-period}

Accident Year Aggregation

Losses are grouped by date of occurrence and calendar year earned
premium is used to approximately match accident year losses.

\section{Advantages}\label{advantages}

\subsubsection{Disadvantages}\label{disadvantages}

Potential mismatch of losses and premiums

Easy to achieve and understand Losses over shorter timeframe than
underwriting year so losses can be reliably estimated sooner Industry
benchmarks are based on AY Valuable when there are economic/regulator
changes or major loss events

\begin{center}\rule{0.5\linewidth}{0.5pt}\end{center}

\subsubsection{Underwriting Year
Aggregation}\label{underwriting-year-aggregation}

Losses are grouped by the year in which the reinsurance policy became
effective (inception date).

\subsubsection{Advantages}\label{advantages-1}

Disadvantagese Extended timeframe for losses tied to UW year Difficult
to isolate the effect of a single large loss event

True match of losses and premiums Valuable when underwriting or pricing.
changes occur

\section{Types of Data}\label{types-of-data}

ULAE - Usually excluded from reinsurance coverage.

ALAE - Generally three possible reinsurance treatments:

·Included with claim amount to determine excess of loss coverage
·Included on a pro-rata basis ( \(\%\) of reinsured loss/total loss)
·Not included in coverage

Multiple currencies -Loss data may be in different currencies.There are
two approaches:

Separate data by currency,then combine the data after translating to a
common currency using the exchange rate at a single point in time
·Aggregate losses are based on the ceding company's currency of origin

Large losses -Exclude large losses from initial projections and then add
in case-specific projections of the reported portion for large losses
and a smoothed provision for the IBNR portion of the large losses.

Recoveries (deductibles, salvage \& subrogation) - Generally recoveries
are applied before reinsurance.

\subsection{Data Challenges for
Reinsurers}\label{data-challenges-for-reinsurers}

·Influence of Change in Operations and Environment

OOperational changes for ceding insurers and reinsurers OChanges in the
legal/economic environment of ceding companies that impact losses

Other Experience Typically Excluded OActuary may exclude discontinued
business (run-off) from the data analysis ·Reporting Lags OLosses first
reported to ceding company before they're reported to the reinsurer 0
Losses may not be reported to the reinsurer until a claim hits a certain
threshold .Heterogeneity of Contract Wording oThe ``manuscript nature''
of reinsurance contracts means contract wording can be different by
contract making it more difficult to aggregate similar data

\begin{center}\rule{0.5\linewidth}{0.5pt}\end{center}

External Sources of Data

Actuaries at smaller reinsurers often use external data sources to help
analyze development or tail factors trend rates,expected loss
ratios,etc.Below are some external sources of data

Reinsurance Association of America (RAA) :Best's Aggregates \& Averages
·Reports from global brokers or reinsurers Other internet searches

External data may be misleading or irrelevant due to differences in:

Manuscript wording/term of reinsurance contractse Mix of assumed
business (differences in industry, region, attachment points, policy
limits) .Types of reinsurance (treaty vs.~facultative, proportional
vs.~non-proportional) Underwriting processes Claims management
differences ·Coding and IT system differences

\subsection{Methods Used}\label{methods-used}

\subsubsection{Development Method}\label{development-method}

·Assumes future development is like prior years' development and losses
to date are predictive of losses yet to be observed Assumes consistency
across experience period of claims processing, mix of business, policy
limits, reinsurance coverage, etc.

\subsubsection{Expected Method}\label{expected-method}

Applies an expected loss ratio to earned premium to estimate expected
losses

Often used when:

·Entering a new line of business or region ·When historical losses are
irrelevant to project future losses due to changes in
strategy,operations, or environment When the development method isn't
appropriate for immature periods because development factors are too
highly leveraged When data are unavailable for other methods

\subsubsection{Bornhuetter-Ferquson
Methoc}\label{bornhuetter-ferquson-methoc}

·Assumes unreported (or unpaid) loss will developbased on expected loss.

\begin{center}\rule{0.5\linewidth}{0.5pt}\end{center}

·Relies on both a selected loss development pattern and expected loss
estimate

Differences in Method Assumptions for Reinsurance vs Primary Insurance

·For a similar line: LDFs at immature ages are often higher (more
leveraged) for reinsurance due to reporting lags ·Loss trend factors are
often higher for excess ofloss reinsurance than primary insurance ·Less
precision in premium on-level factors for rate changes for reinsurance
vs.~primary insurance ·Limited use of adjustment factors for
tort/product reform for reinsurance vs primary insurance

Effect of Changes in Currency Exchange Rates

Many global reinsurers review triangles at the prevailing exchange rates

This prevents distortions in age-to-age factors due to fluctuating
exchange rates year-to-year

Comparisons of Development Factors and Patterns

Reinsurance vs.Primary Insurance (Similar Type of Business)

More volatility in age-to-age factors at earlier maturities for:

Reinsurance compared to primary insurance ·Paid losses compared to
reported losses

Greater volatility means more uncertainty in age-to-age factor selection
and projected ultimate losses

Longer reporting/payment patterns for reinsurance due to lags in
reporting to the reinsurer

Proportional vs.~Non-Proportional Reinsurance (Same Line of Business)

For the same line of business

Significantly more volatility in age-to-age factors for non-proportional
treaty and facultative reinsurance than for proportional treaty
reinsurancee CDFs are greater for non-proportional treaty and
facultative reinsurance (longer dev. patterns Also - More volatility in
ratios of paid-to-reported losses for non-proportional and facultative

A key difference is that proportional reinsurance is ground-up and
non-proportional is excess of loss

Takeaway: The greater volatility in age-to-age factors and diagnostics
results in greater uncertainty in projected ultimate loss from the
development method.

Property Reinsurance excl.Catastrophe vs.Property Reinsurance
Catastrophe

Reinsurer carried reserves for catastrophe losses are usually based on
ground-up exposure-based assessments using info from ceding companies by
contract (NOT the development method)

\begin{center}\rule{0.5\linewidth}{0.5pt}\end{center}

The impact of catastrophe events at different times of the year impacts
development and age-to age factors for different years ODevelopment
method assumptions may not be appropriate .Volatility is much higher for
catastrophe vs.~ex-catastrophe property reinsurance.

Takeaway: Methods using age-to-age factors are often not appropriate for
property catastrophe

\subsection{Implications of Volatility in Loss
Development}\label{implications-of-volatility-in-loss-development}

Greater volatility in age-to-age factors can lead to greater uncertainty
in projections of ultimate loss (and therefore unpaid loss)because
ODevelopment methods and other methods rely on the age-to-age factors
that have greater volatility (B-F method does and the expected method
often does to set ELRs) OGreater volatility in indicated ultimate loss
ratios is often used to select ELRs for the expected and B-Fmethods

Takeaway: Greater volatility in projected ultimate loss results in a
greater range in indicated IBNR and Total Unpaid for:

:Reinsurance compared to primary insurance Non-proportional treaty and
facultative compared to proportional treaty reinsurance Catastrophe
reinsurance compared to excluding catastrophe reinsurance for property

\subsection{Quota Share and Stop Loss Reinsurance
Examples.}\label{quota-share-and-stop-loss-reinsurance-examples.}

\subsubsection{Quota Share Reinsurance}\label{quota-share-reinsurance}

Apply the \%ceded (or 1 - \%retained) to gross ultimate loss, case
reserves, paid loss, and IBNR to get the ceded values. ·The \%ceded may
change over time, so calculations are applied by year

\subsection{Stop-Loss Reinsurancee}\label{stop-loss-reinsurancee}

Stop-Loss reinsurance applies after all other reinsurance and protects
the net result of a ceding company Once a ceding company breaches
stop-loss coverage, the reinsurer will often increase the price or the
attachment point (or both

\begin{center}\rule{0.5\linewidth}{0.5pt}\end{center}

\subsection{LDF Curve-Fitting and Stochastic
Reserving}\label{ldf-curve-fitting-and-stochastic-reserving}

\subsubsection{Overview}\label{overview-4}

Clark is a calculation-heavy paper and you should be able to do the
Variance of Reserve calculations effortlessly by the time of the exam, a
problem which comes up regularly.You should also be prepared fon
questions about model assumptions, diagnostic graphs and how to evaluate
results for reasonableness.

The focus of Clark is to create develop a loss reserving model that
estimates a central loss reserve estimate (with the LDF or Cape Cod
method) and can calculate the variance of the reserve estimate

\subsubsection{Goals of a loss reserving
model}\label{goals-of-a-loss-reserving-model}

Mathematically describe loss emergence to estimate loss reserves

·Estimate the reserve range around the expected reserve,due to variance
from:

OProcess variance - uncertainty due to randomness OParameter variance -
uncertainty in expected value

\subsection{Expected Loss Emergence}\label{expected-loss-emergence}

Instead of using LDFs directly,we're going to fit a curve, \(G(x)\), to
incremental losses using MLE in order to get the best-fitting
parameters. Then, we'll use this curve to estimate the payment pattern
(from 096 to \(100\%\) )as an accident year matures.

Clark uses two different curves,the Loglogistic and Weibull curve.The
Weibull curve has a lighter tail

\pandocbounded{\includegraphics[keepaspectratio]{https://storage.simpletex.cn/view/fabGlAWtl0na5V88igFIrQ1VUqA4sZ1fX}}

\pandocbounded{\includegraphics[keepaspectratio]{https://storage.simpletex.cn/view/f9mKGVc5mle0EEWQG81Q5mopsgGuIFma8}}

\(X\) -average age ofloss occurrence.

Advantages of using a parameterized curve for the loss emergence pattern

·Estimating unpaid losses is simplified (only need 2 parameters)

Can alsouse data that's notfrom a trianglewith evenly spaced evaluation
dates

Payout pattern, \(G(x)\), is a smooth curve and doesn't overfit like
age-to-age factors might

\subsection{Advantage of using data in a tabular
format.}\label{advantage-of-using-data-in-a-tabular-format.}

·Can use data at irregular evaluation periods or when you only don't
have the full triangle

\begin{center}\rule{0.5\linewidth}{0.5pt}\end{center}

\subsubsection{Process Variance}\label{process-variance}

Process variance is the variance due to randomness in the insurance
process

\subsection{Loss Model Assumptions}\label{loss-model-assumptions}

Assume incremental losses have a constant variance/mean ratio,
\(\sigma^{2}\) Assume incremental losses follow an over-dispersed
Poisson model

\subsubsection{Fittinq the loss emerqence
curve}\label{fittinq-the-loss-emerqence-curve}

We find the best-fitting curve using the maximum likelihood method. For
a given set of parameters, we calculate the expected incremental losses,
\(\mu_{i}\) ,for each cell of the loss triangle. Then, we compare the
likelihood that the actual incremental losses came from an
over-dispersed Poisson distribution with the parameters \(\mu_{i}\) and
\(\sigma^{2}\)

The goal is to find the set of parameters that maximize the
log-likelihood function.See the ``Finding BestFit Parameters with MLE''
recipe for an example of how to do this.

Advantages of usinq an over-dispersed Poisson distribution to model
incremental losses

·Using a scaling factor, \(\sigma^{2}\) , allows us to match the
\(1^{\mathrm{st}}\) and \(2^{\mathrm{nd}}\) moments of other
distributions ·TheMaximum Likelihood Estimate reproduces LDF and Cape
Cod loss reserve estimates

\subsubsection{Parameter Variance}\label{parameter-variance}

Parameter variance is the variance in the estimate of the
parameters.It's calculated based on the RaoCramer lower-bound
approximation, using the second derivative information matrix. The
information matrix is used to calculate the covariance matrix,which is
used to calculate the parameter variance. The actual calculation of
theparameter variance is too complexfor the exam

\subsection{Key Model Assumptions}\label{key-model-assumptions}

\subsubsection{.Incremental losses are
iid}\label{incremental-losses-are-iid}

OIndependent - One period doesn't impact surrounding periods (this
assumption fails if there are calendar year effects such as inflation)
OIdentically distributed - Assume the same emergence pattern, \(G(x)\)
,for all accident years (this assumption fails if the mix of business or
claims handling changes)

:Variance/Mean Scale parameter, \(\sigma^2\) ,is fixed and known

oWe ignore the variance of \(\sigma^{2}\)

.Variance estimates use approximation to the Rao-Cramer lower bound

Themodel assumptionsmean there's a potential thatfuture losseshave
higher variance than what the model indicates.

\begin{center}\rule{0.5\linewidth}{0.5pt}\end{center}

\subsection{LDF Method}\label{ldf-method}

A problem with the LDF method is that there is a parameter for each
accident year for the current LossAy See the ``Variance of Reserves (LDF
Method)'' recipe for how to calculate reserve variance.

\section{CapeCod Method}\label{capecod-method}

The Cape Cod method uses additional information, an exposure base. Clark
recommends on-level earned premium, but another exposure base can be
used as long as it's proportional to ultimate expected losses by
accident year.

Premium must be on-leveled so that we can assume a constant ELR across
all accident years. We could also adjust for loss trend net of exposure
trend so that all accident years are at the same cost level.

\subsubsection{LDF Method vs.Cape Cod
Method}\label{ldf-method-vs.cape-cod-method}

LDF method over-parameterizes the model, fits to the noise'' in the data

OFor a 10-yr triangle, there are 12 parameters to estimate and only 55
data points

·Cape Cod has lower parameter variance because there arefewer parameters
to estimate and it uses more information (on-level premium)

\subsubsection{Variance of Loss
Reserves}\label{variance-of-loss-reserves}

Once you have the loss reserve estimate, the scale parameter and the
parameter variance,you can calculate the variance around theloss reserve
estimate

\[ProcessVar=\sigma^{2}\cdot Resv\]

TotalVariance = ProcessVariance+ ParameterVariance

\subsection{Diagnostics}\label{diagnostics}

ELR for Cape Cod

The Cape Cod method assumes a constant ELR across all accident years.
Test this assumption by graphing the estimated ultimate loss ratios by
accident year.

The estimated ultimate loss ratios should be random around the ELR with
no patterns or trends.If there is a pattern, then the assumption of a
constant ELR isn't reasonable and the Cape Cod reserve estimate could be
biased.

\pandocbounded{\includegraphics[keepaspectratio]{https://storage.simpletex.cn/view/fqvsNkgO2U0zuAVMgf5GIqCNP4UQpFmA2}}

\subsection{Residual Graphs}\label{residual-graphs}

\pandocbounded{\includegraphics[keepaspectratio]{https://storage.simpletex.cn/view/f3Gq6KCFdS8nEstDSb2LTgrOULgsiv2bp}}

\(r_{Y,k} = \frac{InCloss_{Y,k}}{\sqrt{\sigma^2 + \mu_{Y,k}}}\)

\begin{center}\rule{0.5\linewidth}{0.5pt}\end{center}

Residual graphs are an important diagnostic to test the underlying
assumptions.Below are the graphs Clark specifically mentions and what
assumptions they test:

·Normalized Residuals vs.~Increment Age OTests how well the loss
emergence curve fits incremental losses at different development
periods. :Normalized Residuals vs.~Expected Incremental Loss (
\(\mu_{i}\) ) OTests the variance/mean ratio \(\sigma^{2}\colon\) If the
variance/mean ratio is not constant, we should see residuals clustered
closer to zero at either high or low expected incremental losses
Normalized Residuals vs.~Calendar Year OTests whether there are diagonal
effects (e.g.~high inflation in a calendar year)

For all the graphs, the residuals should be random around zero with no
patterns or autocorrelations. If this isn't the case,then some
assumptions of the model areincorrect

\subsubsection{Process vs.Parameter
Variance}\label{process-vs.parameter-variance}

We should see that parameter variance is greater than process variance.
This is because most of the uncertainty is due to the inability to
estimate the expected reserve (parameter variance)rather than
uncertainty due to random events (process variance)

The reason for this is that there are so few data points in a loss
triangle to estimate the parameters.The Cape Cod method lowers the
parameter variance by including the exposure data

\subsubsection{Other Calculations}\label{other-calculations}

\subsection{Variance of Prospective
Losses}\label{variance-of-prospective-losses}

The regular Clark method calculates reserve variance for past accident
years. This same approach can be used to calculate variability around
prospective losses for the prospective accident year (period). See the
``Variance of Prospective Losses'' recipe for this calculation.

\subsubsection{Calendar Year
Development}\label{calendar-year-development}

Instead of estimating the total unpaid loss reserve, we can calculate
the unpaid losses that we expect will be paid over the next calendar
year and create a range around that.

The advantage of this type of calculation is that the model can be
tested in a relatively short period of time. After one year, the actual
12-month loss development can be compared to the original forecasted
range to test whether the actual development falls within the range.See
the``Variance of Calendar Year Development'' recipe for this
calculation.

\subsection{Variance of Discounted
Reserves}\label{variance-of-discounted-reserves}

The discounted paid loss reserve is calculated by discounting the future
payments at the half-year mark.For the exam a calculation problem
doesn't seem testable

\begin{center}\rule{0.5\linewidth}{0.5pt}\end{center}

Onekey point is that the CV of the discounted loss reserve is smaller
than the CV of the undiscounted loss reserve.This is because the tail of
the payout curve has the greatest parameter variance,but is discounted
the most.

\subsection{Adjustments for Other Exposure
Periods}\label{adjustments-for-other-exposure-periods}

The formula for \(G(x)\) is only valid on an accident year basis where
the first development period is at 12 months. If the first development
period is shorter than 12 months or policy year is used, then you need
to make some adjustments.

Percent of ultimate loss as of time \(t\), with annualization

\[G(t)=expos(t)\cdot G(x)\quad\mathrm{where}\quad x=AvgAge(t)\]

\begin{longtable}[]{@{}
  >{\raggedright\arraybackslash}p{(\linewidth - 4\tabcolsep) * \real{0.0346}}
  >{\raggedright\arraybackslash}p{(\linewidth - 4\tabcolsep) * \real{0.2962}}
  >{\raggedright\arraybackslash}p{(\linewidth - 4\tabcolsep) * \real{0.6692}}@{}}
\toprule\noalign{}
\begin{minipage}[b]{\linewidth}\raggedright
\end{minipage} & \begin{minipage}[b]{\linewidth}\raggedright
Accident Year
\end{minipage} & \begin{minipage}[b]{\linewidth}\raggedright
Policy Year
\end{minipage} \\
\midrule\noalign{}
\endhead
\bottomrule\noalign{}
\endlastfoot
expos(t) &
\(\\begin{cases} \\frac{t}{12}, & t \\leq 12 \\\\ 1, & t > 12 \\end{cases}\)
&
\(\\begin{cases} \\frac{1}{2}\\left(\\frac{t}{12}\\right)^2, & t \\leq 12 \\\\ 1-\\frac{1}{2}\\left(2-\\frac{t}{12}\\right)^2, & 12 < t \\leq 24 \\\\ 1, & t > 24 \\end{cases}\) \\
AvgAge(t) &
\(\\begin{cases} \\frac{t}{2}, & t \\leq 12 \\\\ t - 6, & t > 12 \\end{cases}\)
&
\(\\begin{cases} \\frac{t}{3}, & t \\leq 12 \\\\ \\frac{(t-12)+\\frac{1}{2}(24-t)(1-expos(t))}{expos(t)}, & 12 < t \\leq 24 \\\\ t-12, & t > 24 \\end{cases}\) \\
\end{longtable}

I would know all the formulas for accident year.For policy year, I would
make sure to know the special cases where \(t=12\) months and where
\(t=24\)

\begin{longtable}[]{@{}lll@{}}
\toprule\noalign{}
Policy Year & expos(t) & AvgAge(t) \\
\midrule\noalign{}
\endhead
\bottomrule\noalign{}
\endlastfoot
t = 12 & 50\% & 4 \\
t = 24 & 100\% & 12 \\
\end{longtable}

\subsection{Recipes for Calculation
Problems}\label{recipes-for-calculation-problems-3}

Variance of Reserves (LDF Method) Variance of Reserves (Cape Cod Method)
Normalized Residuals Variance of Prospective Losses :Variance of
Calendar Year Development Finding Best-Fit Parameters with MLE

\begin{center}\rule{0.5\linewidth}{0.5pt}\end{center}

\subsection{Measuring the Variability of Chain Ladder Reserve
Estimates}\label{measuring-the-variability-of-chain-ladder-reserve-estimates}

\subsection{Key Ideas}\label{key-ideas}

The goal of this paper is to use the chain ladder method to create a
confidence interval around the ultimate loss and estimatedloss reserve

Mack-Chain Ladder focuses on two main ideas:

1.Three main assumptions underly the chain ladder method.For a loss
triangle,we should look at different diagnostic graphs and tests to see
whether the chain ladder method is appropriate or not.

2.Based on the three assumptions,we can calculate the estimated loss
reserve and the standard error of the estimated loss reserve for each
accident year and for all years combined with the chain ladder method.
Then, after making an assumption about the distribution of the loss
reserve (e.g.~lognormal) we can calculate loss reserve confidence
intervals.

\subsection{Mack Assumptions}\label{mack-assumptions}

1.Expected losses in the next development period are proportional to
losses-to-date

\[\mathrm{E}\Big[C_{i,k+1}\:|\:C_{i,1},\cdots,C_{i,k}\Big]=C_{i,k}\cdot LDF\]

·The chain ladder method uses the same LDF for each accident year

Uses most recent losses-to-date to project losses, ignoring losses as of
earlier development periods

\begin{enumerate}
\def\labelenumi{\arabic{enumi}.}
\setcounter{enumi}{1}
\item
  Losses are independent between accident years.
\item
  Variance of losses in the next development period is proportional to
  losses-to-date with proportionality constant \(\alpha_{k}^{2}\) that
  varies by age.
  \[\mathrm{Var}\Big[C_{i,k+1}\:|\:C_{i,1},\cdots,C_{i,k}\Big]=C_{i,k}\cdot\alpha_{k}^{2}\]
\end{enumerate}

\subsection{Alternative Variance Assumptions (Weighted
LDFs)}\label{alternative-variance-assumptions-weighted-ldfs}

The chain ladder method in Mack - Chain Ladder uses volume-weighted
LDFs. This implies that variance of losses in the next development
period \((L_{oss_{k+1}})\) is proportional to losses-to-date
\((Loss_{k})\) Assumption 3.

If we calculate the LDFs a different way, such as the simple average of
the age-to-age factors or age-to-age factors weighted by \(Loss_k^2\)
,then we're making a different variance assumption Variance Assumptions

\begin{longtable}[]{@{}lll@{}}
\toprule\noalign{}
Var\((C_{i,k}) \propto\) & LDF calc. & LDF \\
\midrule\noalign{}
\endhead
\bottomrule\noalign{}
\endlastfoot
1 & \(C^2_{i,k}\) & \(C^2_{i,k}\) -wtd \\
\(C_{i,k}\) & \(C_{i,k}\) & Vol-weighted \\
\(C^2_{i,k}\) & 1 & Simple Avg \\
\end{longtable}

\begin{center}\rule{0.5\linewidth}{0.5pt}\end{center}

\subsubsection{Variance of Reserves}\label{variance-of-reserves}

Using the three chain ladder assumptions, we can calculate the variance
of the reserve estimate with Mack's formula for the standard error of
the reserve. We get the estimated reserve and the standard error of the
estimate by accident year and for the overall reserve

One disadvantage of the Mack method (compared to the bootstrap method)
is that it doesn't tell us about the shape of the loss reserve
distribution. It only gives us the mean and standard deviation. To
create confidence intervals,we need to make an assumption about the
distribution.

\subsubsection{Reserve Confidence
Intervals}\label{reserve-confidence-intervals}

Once we've done all the calculations, we have an expected loss reserve
estimate (the loss reserve estimate from the standard chain ladder
method) and the standard error (standard deviation) of the loss reserve
estimate.Besides these values, the method doesn't tell us anything else
about the distribution of the loss reserve estimate

So, we're going to assume a distribution and use the mean and standard
deviation of the loss reserve that we calculated. Mack looks at two
distributions: Normal and Lognormal

Normal Distribution

Lognormal Distributior

\pandocbounded{\includegraphics[keepaspectratio]{https://storage.simpletex.cn/view/fMXxRe4VXrUsiC18VIoVlunkRQ2RqfzDC}}

\pandocbounded{\includegraphics[keepaspectratio]{https://storage.simpletex.cn/view/f7Isbf4yYS4RIGqADqGaCHkYzFyRfC49v}}

If min of C.I. is negative OR s.e. \(R)>50\%\) of \(R\) use Lognormal

\subsection{Checking the Chain Ladder
Assumptions.}\label{checking-the-chain-ladder-assumptions.}

The three Mack assumptions have significant implications, so we should
take a look at some different tests and diagnostics to see how well the
assumptions hold up.

\subsubsection{Plot of Cumulative Losses from Adjacent
Periods}\label{plot-of-cumulative-losses-from-adjacent-periods}

This plot tests assumption 1. We want to see if losses at the next
period are proportional to losses-to-date with no intercept.

\subsection{If the assumption holds, you should
see:}\label{if-the-assumption-holds-you-should-see}

Linear relationship between. \(Loss_{k+1}\) and Loss, through the origin
(no intercept)

\pandocbounded{\includegraphics[keepaspectratio]{https://storage.simpletex.cn/view/fnuOFAuKbTbrVVWYPlPkDagz8QvhbGDX5}}

·Line should go through the data points

\subsection{If the assumption is violated,you might
see}\label{if-the-assumption-is-violatedyou-might-see}

·The data points show that there should be anintercept term in the
regression line .The relationship isn't linear

\begin{center}\rule{0.5\linewidth}{0.5pt}\end{center}

\subsubsection{Plot of Weighted
Residuals}\label{plot-of-weighted-residuals}

This plot primarily tests assumption 3, the variance assumption

\pandocbounded{\includegraphics[keepaspectratio]{https://storage.simpletex.cn/view/fgMvNrsPbtdppGmfLT4qX0P3hht2IHHHS}}
010,00Doss20,000 30,000

\subsection{If the assumption holds, you should
see:}\label{if-the-assumption-holds-you-should-see-1}

Residuals should be random around zero with no significant trends or
patterns.

\subsection{If the assumptionis violated, you might
see:}\label{if-the-assumptionis-violated-you-might-see}

·Variance of residuals is higher at one end of the graph than at the
other Residuals show an increasing (or decreasing trend)

If the residuals for a few development periods aren't random, we can
graph the residual plots using the LDFs for the alternative variance
assumptions: \(\operatorname{Var}(C_{k+1})\propto C_{k}^{2}\) and
Var\((C_{k+1})\propto1\) If the residual plots for one of the
alternative assumptions is more random, then we could replace the
volume-weighted LDF with the alternative LDF( \(f_{\mathrm{ko}}\) or
\(f_{\text{k}2}\) ).

\subsubsection{Testing Assumption 2- Independence between accident years
(Calendar Year
Test}\label{testing-assumption-2--independence-between-accident-years-calendar-year-test}

The assumption ofindependence between accident years implies that there
are no calendar year effects that impact losses from multiple accident
years (causing a dependence between accident years)

We test the null hypothesis that there are no calendar year effects with
Mack's calendar year test (see the Mack - Chain Ladder Calendar Year
Test'' recipe). If there are significant calendar year effects, then
assumption 2 is violated.

Examples of calendar year effects:

·Major changes in claims handling practices ·Major changes in setting
case reserves ·Unexpectedly high (or low) inflation ·Significant changes
due to court decisions

\subsubsection{Testing for Correlation Between Adjacent LDFs (Assumption
1)}\label{testing-for-correlation-between-adjacent-ldfs-assumption-1}

The first assumption implies that subsequent development factors are
uncorrelated (e.g.~high development from ages 12 to 24 gives no
information about how losses develop from age 24 to 36)

In the chain ladder method, we always use the same LDF from ages 24 to
36 no matter how losses developed from ages 12 to 24. For instance, if
the overall LDF from ages 12 to 24 is 3.5 and a particular AY developed
from 1,000 to 5,500 (age-to-age factor of 5.5), we still expect the same
96 development from 24 to 36.

Example: If a book of business typically shows a smaller-than-average
increase \(Loss_{k+1}/Loss_{k}<LDF_{k}\) , after a larger-than-average
increase, \(Loss_{k}/Loss_{k-1}>LDF_{k-1}\) ,then the chain ladder
method would not be appropriate

\begin{center}\rule{0.5\linewidth}{0.5pt}\end{center}

Mack-ChainLadder uses Spearman's rank correlation to test for
correlation between LDFs(See the ``Mack - Chain Ladder - Correlation of
Adjacent LDFs'' recipe)

The Mack Correlation of Adjacent LDFs test looks at the triangle as a
whole, not at each column-pair separately. This is because it's more
important to know if correlation between LDFs prevails throughout the
triangle

Advantages of using Spearman's rank:

The test is distribution-free (doesn't assume LDFs are from a normal
distribution ·Differences in variances of LDFs between development
periods is less important because it uses ranks

\subsection{Reviewing Results for
Reasonableness}\label{reviewing-results-for-reasonableness}

·MSE of the total reserve is greater than the sum of the MSE from the
reserve for individual accident years. OBecause the same LDFs are used
for all accident years, the loss reserve estimates are positively
correlated between accident years, increasing the total MSE Higher
standard error percentage for the oldest accident years OThe absolute
standard error should be lowest for the oldest accidentyears,but since
the size of the loss reserve is so small, the standard error percentage
will be higher. ·Higher standard error percentage for the most recent
accident year (or two) OUncertainty in forecasting future losses is
highest for the most recent accident years because losses are immature,
so the standard error percentage will be higher.

\subsubsection{Weaknesses of the Chain Ladder
method}\label{weaknesses-of-the-chain-ladder-method}

·Estimators of thelast 2 or3LDFsrely on very few observations Doesn't
work well for the most recent accident year where losses-to-date provide
a very uncertain base to project ultimate losses OAnother method, such
as the least squares method would put less credibility on the immature
losses.

\subsection{Recipes for Calculation
Problems}\label{recipes-for-calculation-problems-4}

Residual Test Calendar Year Test Reserye Confidence Interyal MSE
Calculation Correlation of Adjacent LDFs Overall MSE Calculation

\begin{center}\rule{0.5\linewidth}{0.5pt}\end{center}

\subsection{Testing the Assumptions of Age-to-Age
Factors}\label{testing-the-assumptions-of-age-to-age-factors}

\subsection{Overview}\label{overview-5}

Think of Venter Factors as an extension of the Mack (1994) paper. Venter
takes a look at the three key Mack assumptions underlying the chain
ladder method and identifies six ways to test those assumptions.If those
tests fail for a loss triangle,Venter points out a few alternative loss
reserving methods to use instead.

As you go through Venter Factors, think about how it ties together with
Mack (1994) to give a more complete picture of the chain ladder method
and how to use the chain ladder method to create reserve ranges.

\subsection{MackAssumptions-Restated}\label{mackassumptions-restated}

If the assumptions below hold, then the chain ladder method gives the
minimum variance unbiased linear estimator of future loss emergence.

Assumption 1

Expected incremental loss emergence is proportional to cumulative
losses-to-date

\[\mathrm{E}\Big[\:IncLoss_{AY,k+1}\:|\:data\Big]=\Big(LDF_{k}-1\Big)\cdot Loss_{AY,k}\]

Assumption 2

Losses are independent between accident years

Assumption 3

Variance of the next incremental loss is a function of age and
cumulative losses-to-date.

\[\mathrm{Var}\Big[IncLoss_{AY,k+1}\:|\:data\Big]=function\Big(k,Loss_{AY,k}\Big)\]

We can't directly test these assumptions, but there are six testable
implications that fall out of these assumptions that we can test.If the
tests for any of the implications fails, then the three Mack assumptions
don't hold up, meaning that the chain ladder method doesn't give the
minimum variance unbiased linear estimate of future losses.

\begin{center}\rule{0.5\linewidth}{0.5pt}\end{center}

\subsection{Testable Implication 1 -Significance of development
factors}\label{testable-implication-1--significance-of-development-factors}

Tests Assumption 1: Linear loss emergence proportional to loss-to-date

Assumption 1 assumes incremental losses are proportional to
losses-to-date with a development factor. If that's the case, the
development factor shouldbe statistically significant.

Later in the paper, Venter shows the regression including a constant,
\(\hat{y} = a+ bx\) . This tests the alternative linear with constant
(least squares) model compared to the chain ladder model and also
indirectly tests for implication 1. Excluding the constant term in the
regression will give a more direct test of implication 1.

One last point: You can't just do a regression of \(L_{oss_{k+1}}\)
vs.Loss. This is because what we really want to predict is the
incrementa/loss from one period to the next.If we use \(Loss_{k+1}\),
then comparing the coefficient \(b\) to the standard deviation directly
will overstate the significance of thefactor.

\subsection{Testable Implication 2 - Superiority to alternative loss
emergence
patterns}\label{testable-implication-2---superiority-to-alternative-loss-emergence-patterns}

Tests Assumption 1: Linear loss emergence proportional to loss-to-date

If assumption 1 is correct, then the chain ladder model should predict
incremental losses better than alternative loss emergence patterns. We
can use a goodness-of-fit test to compare the alternative models to the
chain ladder method.

Different Loss Emergence Patterns

\begin{longtable}[]{@{}
  >{\raggedright\arraybackslash}p{(\linewidth - 2\tabcolsep) * \real{0.6104}}
  >{\raggedright\arraybackslash}p{(\linewidth - 2\tabcolsep) * \real{0.3896}}@{}}
\toprule\noalign{}
\begin{minipage}[b]{\linewidth}\raggedright
Expected Incremental Loss is\ldots{}
\end{minipage} & \begin{minipage}[b]{\linewidth}\raggedright
Model
\end{minipage} \\
\midrule\noalign{}
\endhead
\bottomrule\noalign{}
\endlastfoot
\ldots proportional to loss-to-date & Chain Ladder \\
\ldots proportional to loss-to-date with a constant & Least Squares \\
\ldots percentage of ultimate loss & Bornhuetter-Ferguson: f(d)h(w) \\
\end{longtable}

Alternative Pattern 1 - Linear with a constant (least squares method).

\[\mathop{\mathrm{E}}\Big[IncLoss_{d}\Big]=f(d)\cdot Loss_{k}+g(d)\\\mathop{\mathrm{factor}}_{\mathrm{constant}}\]

Including a constant term is often significant in the first development
period or two, especially for highly variable and long-tail lines

Test this alternative pattern by running a linear regression of
\(IncLoss_{k+I}\) Vs. \(Loss_k\) to get the factors of the development
formula \(\hat{y}=a+bx\) If the constant is statistically significant,
this loss emergence pattern is more strongly supported than the
chainladder method

\subsubsection{Alternative Pattern 2-Factor times parameter
(parameterized BF
model)}\label{alternative-pattern-2-factor-times-parameter-parameterized-bf-model}

If expected incremental loss is modeled as a percent of ultimate, we can
use the parameterized BF model

\pandocbounded{\includegraphics[keepaspectratio]{https://storage.simpletex.cn/view/fF8RKSoTaD4T6iNERyvWGXTV8xOviEksK}}

\begin{center}\rule{0.5\linewidth}{0.5pt}\end{center}

We need to solve for the f(d) and h(w) factors iteratively.Note that the
final f(d) factors won't necessarily sum to 1. Likewise, the h(w)
factors aren't equivalent to estimated ultimate loss. Instead, f(d) and
h(w) are used together to estimate future incremental losses.

Compare the parameterized BF model to the chain ladder model by doing a
goodness-of-fit test. If the parameterized BF model has a lower test
statistic, then the loss emergence in the loss triangle is better
described by the parameterizedBF model.

The parameterized BF model has twice as many parameters (2n-2) as the
chain ladder model. So,it's often over-parameterized, resulting in a
higher adjusted SSE. We can improve the performance by reducing the
number of parameters (which will increase SSE, but hopefully decrease
Adjusted SSE))

The parameterized Cape Cod method uses one h parameter for all accident
years. If the ultimate loss for each accident year is roughly the same,
then this model might be an improvement over the fully parameterized BF

This is a bit different than the normal Cape Cod model that uses the
same expected loss ratio for all accident years. A possible improvement
is to use a loss ratio triangle with the Cape Cod method, because a
constant ultimate loss ratio over all accident years is more reasonable

\subsection{Disadvantages of the Cape Cod
method:}\label{disadvantages-of-the-cape-cod-method}

·Requires relatively stable level of loss exposure over accident years

·Need to adjust for known exposure and price level differences between
accident years

\subsection{Testable Implication 3 - Linearity of the model: Residuals
vs.~Lossk.}\label{testable-implication-3---linearity-of-the-model-residuals-vs.-lossk.}

Tests Assumption 1: Linear loss emergence proportional to loss-to-date

Assumption 1 assumes incremental loss emergence is linearly proportional
to loss-to-date.

If residuals show a non-linear pattern (e.g.~positive - negative -
positive), then incremental loss emergence is a non-linear function of
loss-to-date

\subsection{Testable Implication 4 - Stability of development factors:
Residuals
vs.~time}\label{testable-implication-4---stability-of-development-factors-residuals-vs.-time}

Tests Assumption 1: Linear loss emergence proportional to loss-to-date

Assumption 1 uses the same development factor for all accident years, so
we would expect that the appropriate development factor is stable over
time.

A pattern of high/low residuals when plotted against time indicates
instability and is evidence against using the same development factor
for all AYs.

If the age-to-age factors appear stable, we should use all accident
years when calculating the LDFs (to minimize variance). But, if it looks
like the average level of the age-to-age factors is shifting over time,
we could use a weighted average of the LDFs over the last n-years
(e.g.~5-year weighted average LDFs)

\begin{center}\rule{0.5\linewidth}{0.5pt}\end{center}

\subsubsection{Testable Implication 5 - Correlation of development
factors}\label{testable-implication-5---correlation-of-development-factors}

Tests Assumption 2: Independence between Accident Years *

If there is significant correlation between development factors at
different ages, then this is evidence against the chain ladder method.

\subsubsection{*Confliction with Mack
(1994):}\label{confliction-with-mack-1994}

There's some overlap between Mack assumptions 1 and 2 as they're
discussed in the two papers:

●In Mack (1994) (pg. 109), Mack states that the linearity assumption
(assumption 1) implies that ``subsequent development factors\ldots{} are
uncorrelated.'' In Venter Factors (pg.832-833), Venter says that the
correlation test is to check for dependencies in the triangle,which is a
test on theindependence assumption (assumption 2)

Just note that there's some disagreement between Venter and Mack
regarding the assumption tested by correlation of development columns

\subsection{Testable Implication 6 - Significantly high or low diagonals
(calendar year
effects)}\label{testable-implication-6---significantly-high-or-low-diagonals-calendar-year-effects}

Tests Assumption 2: Independence between Accident Years

Run a regression of incremental losses against cumulative losses at the
prior development periods and include a dummy variable for each
diagonal.

If any of the dummy variables are statistically significant, then this
indicates calendar year effects.

\subsection{Recipes for Calculation
Problems}\label{recipes-for-calculation-problems-5}

\begin{description}
\item[Correlation of Development Factors]
Parameterined Br: \(f(d)h(w)-\)Var c\(f(d)h(w)\) \(oc\) sta ce
\end{description}

Testing for Significantly High/Low Diagonals Goodness of Fit Test

\begin{center}\rule{0.5\linewidth}{0.5pt}\end{center}

Using the ODP Bootstrap Model

\subsubsection{Overview}\label{overview-6}

The goal of the bootstrap model is to create a full distribution of
possible loss reserve outcomes, not just a point estimate or the mean
and variance like in the Mack (1994) approach.It does this by sampling
residuals from the original loss triangle to create simulated loss
triangles, one for each iteration in the simulation. Then, an estimated
loss reserve is calculated for each iteration.

\subsection{GLM Bootstrap Model}\label{glm-bootstrap-model}

The recipes in the Exam 7 Cookbook show the different types of problems
you might be asked from the bootstrap model. Below is a high-level
overview of how the basic GLM bootstrap model works.

\subsubsection{GLM Setup}\label{glm-setup}

1)Shapland uses a GLMwith a log-link function and over-dispersed Poisson
error distribution

Variance of Incremental Losses

\[\mathrm{Var}\Big[q(w,d)\Big]=\phi\cdot m_{\omega,d}^{z}\]

Expected Incremental Losses

\[
\begin{aligned}
&\operatorname{E}\Bigl[q(w,d)\Bigr]\:=\:m_{_{\infty,d}} \\
&\ln\left(m_{_{\omega,d}}\right)=\eta_{_{\omega,d}} \\
&\eta_{\omega,d}=\alpha_{\omega}+\sum_{d=2}\beta_{d}
\end{aligned}
\]

2)Next, set up the design matrix and the parameters for the model. The
basic ODP bootstrap has a parameter for each accident year and
development period. The GLM framework is flexible, so we can group some
of those parameters together or add parameters for diagonal effects.

Y -Vector of the log-link triangle X -Design Matrix to solve the
specified model A -Vector of the \(\alpha\) and \(\beta\) model
parameters

Y = X \times A

\$

\begin{bmatrix}
\ln(g(1,1)) \\
\ln(g(2,1)) \\
\ln(g(3,1)) \\
\ln(g(1,2)) \\
\ln(g(2,2)) \\
\ln(g(1,3))
\end{bmatrix}

=

\begin{bmatrix}
1 & - & - & - \\

- & 1 & - & - \\
- & - & 1 & - \\
  1 & - & - & 1 \\
- & 1 & - & 1 \\
  1 & - & 1 & 1
  \end{bmatrix}
  \times
  \begin{bmatrix}
  \alpha_1 \\
  \alpha_2 \\
  \alpha_3 \\
  \beta_1 \\
  \beta_2 \\
  \beta_3
  \end{bmatrix}

\$

Solution Matrix = X \times A

\$

\begin{bmatrix}
\ln(m*{1,1}) \\
\ln(m*{2,1}) \\
\ln(m*{3,1}) \\
\ln(m*{1,2}) \\
\ln(m*{2,2}) \\
\ln(m*{1,3})
\end{bmatrix}

=

\begin{bmatrix}
1 & - & - & - \\

- & 1 & - & - \\
- & - & 1 & - \\
  1 & - & - & 1 \\
- & 1 & - & 1 \\
  1 & - & 1 & 1
  \end{bmatrix}
  \times
  \begin{bmatrix}
  \alpha_1 \\
  \alpha_2 \\
  \alpha_3 \\
  \beta_1 \\
  \beta_2 \\
  \beta_3
  \end{bmatrix}

\$

\begin{center}\rule{0.5\linewidth}{0.5pt}\end{center}

3)Once the model is all set up, fit the model to the triangle of
incremental losses using iterated least squares or maximum likelihood.
This gives us the best-fitting parameters.

4)Using the best-fitting parameters, calculate the fitted incremental
loss triangle.

\[\ln\bigl(m_{\omega,d}\bigr)=\eta_{\omega,d}=\alpha_{\omega}+\sum_{d=2}\beta_{d}\quad m_{\omega,d}=e^{n_{\omega,d}}\]

5)Now, calculate the unscaled Pearson residuals between the original
triangle of actual incremental losses and the triangle of fitted
incremental losses.

\[r_{\omega,d}=\frac{q(w,d)-m_{\omega,d}}{\sqrt{m_{\omega,d}^{z}}}\]

\begin{enumerate}
\def\labelenumi{\arabic{enumi}.}
\setcounter{enumi}{5}
\tightlist
\item
  Use the unscaledPearson residuals to calculate the scale parameter.
\end{enumerate}

\[\phi=\frac{\sum r_{\omega,d}^{2}}{N-p}\quad\begin{array}{c}N\to\#\text{incremental values}\\p\to\#\text{parameters}(\alpha's,\beta's,\text{hetero-adj parameters})\end{array}\]

7)Calculate the standardized Pearson residuals by applying the Hat
Matrix adjustment factor to the unscaled Pearson residuals,

\[r_{w,d}^{H}=r_{w,d}\cdot f_{w,d}^{H}\quad\mathrm{where}\quad f_{w,d}^{H}=\sqrt{\frac{1}{1-\mathbf{H}_{w,d}}}\]

\subsection{Running the Bootstrap
Model}\label{running-the-bootstrap-model}

8)For each iteration, sample with replacement from the standardized
Pearson residuals and calculate the sampled loss triangle.
\[q^{*}(w,d)=r^{*}\cdot\sqrt{m_{\omega,d}^{*}}+m_{\omega,d}\]

9)For each iteration, refit the GLM to the incremental losses to get
new parameters, specific to this iteration's sampled loss triangle.

\begin{enumerate}
\def\labelenumi{\arabic{enumi}.}
\setcounter{enumi}{9}
\tightlist
\item
  Use the new parameters to calculate the projected unpaid incremental
  losses.
\end{enumerate}

\[\ln\left(m_{\omega,d}^{iter}\right)=\alpha_{\omega}^{iter}+\sum_{d=2}\beta_{d}^{iter}\]

\begin{enumerate}
\def\labelenumi{\arabic{enumi}.}
\setcounter{enumi}{10}
\tightlist
\item
  Add process variance to the projected unpaid incremental losses with
  the Gamma distribution.
\end{enumerate}

\[q^{im}(w,d)\sim\mathrm{Gamma}\Big(\mathrm{mean}=m_{w,d}^{iter},\mathrm{var}=\phi\cdot m_{w,d}^{iter}\Big)\]

\subsubsection{Shapland Bootstrap Excel
Files}\label{shapland-bootstrap-excel-files}

Ihighly recommend thatyou spend some time to see how theExcelfiles
referenced in the syllabus that accompany the Shapland paper work.It
will give you a much clearer idea of how the bootstrap model works. I
particularly recommend looking at the ``GLM Framework 6.xlsm'' file.

\subsubsection{Advantages of the GLM
Eramework}\label{advantages-of-the-glm-eramework}

·Can tailor the model to the statistical features of the data (e.g.~can
add a diagonal parameter, \(\gamma_{k}\) )

\begin{center}\rule{0.5\linewidth}{0.5pt}\end{center}

Can use fewer parameters to avoid over-parameterization (e.g.~fewer AY
parameters.

Can model data that's not in a loss triangle (e.g.~missing data for the
first few diagonals)

Disadvantages of the GLM Framework

Simulation is slower because the GLM must be solved for in each
iteration

·Can't directly explain the model using LDFs

\subsection{ODP Bootstrap Model (Simplified
GLM)}\label{odp-bootstrap-model-simplified-glm}

The simplified GLM approach can be used if the GLM bootstrap model has a
parameter for each accident year and development period and uses an
over-dispersed Poisson error distribution.

Instead ofusing the GLM to calculate the parameters,use the all-years
volume-weighted LDFs to calculate expected incremental losses,
\(m_{w,d\bullet}\) After creating the sample loss triangles using
resampled residuals, calculate new LDFs and use them to calculate the
projected unpaid incremental losses. Process variance is incorporated
the same way with the Gamma distribution.

\subsubsection{Advantages of the ODP Bootstrap (Simplified
GLM}\label{advantages-of-the-odp-bootstrap-simplified-glm}

·Can use the simpler LDF method and the model will still be based on the
GLM framework ·Using LDFs makes the model more easily explainable to
others ·The GLM uses a log-link and may not work with negative
incrementals, but the simplified GLM will still get a solution

\subsubsection{Disadvantages of the ODP Bootstrap (Simplified
GLM)}\label{disadvantages-of-the-odp-bootstrap-simplified-glm}

Unable to adjust for calendar-year effects ·Requires many parameters and
can over-fit the data

\subsection{Practical Issues}\label{practical-issues}

Negative Incremental Values

Because a log-link is used, the GLM framework requires incremental
losses to be greater than zero

Adjustment 1:Sum of column ofincremental losses is positive

Calculate the log-link triangle but adjust the calculation for negative
incremental losses to \(-\ln\left[abs(IncLoss)\right]\)

Adjustment 2: Sum of column of incremental losses is negative

Shift allthe incremental losses \(up\) by the size of the largest
negative incremental ·Run themodel Shift all the fitted incremental
losses back down

\begin{center}\rule{0.5\linewidth}{0.5pt}\end{center}

\subsection{\texorpdfstring{Negative values during simulation of process
variance: An issue that can ariseis if the model simulates negative
incremental losses, \(m_{w,d}.\) This is a problem
because}{Negative values during simulation of process variance: An issue that can ariseis if the model simulates negative incremental losses, m\_\{w,d\}. This is a problem because}}\label{negative-values-during-simulation-of-process-variance-an-issue-that-can-ariseis-if-the-model-simulates-negative-incremental-losses-m_wd.-this-is-a-problem-because}

the process variance step can't handle negative incremental losses
because the variance of the Gamma distribution would be negative.

Shapland shows two possible ways to adjust themodel to handle negative
\(m_{w,d}\) values during simulation on Pg. 21-22.For both options, a
value is simulated from a modified version of the Gamma distribution,
using the absolute value of the negative expected loss \(m_{w,d}\) .Both
options transform the distribution so that the simulated value has an
expected value equal to the negative incremental loss \(m_{w,d}\)

Below is a graphical view of how these two adjustments work:

Distribution based on the absolute value of \(m_{\omega,d}\)
\[\mathrm{Gamma}\Big(\mathrm{mean}=\Big|m_{\omega,d}\Big|,\mathrm{var}=\phi\cdot\Big|m_{\omega,d}\Big|\Big)\]

\pandocbounded{\includegraphics[keepaspectratio]{https://storage.simpletex.cn/view/fn71mTIbQv0ui5uhrvgKOOFcx8NGRMD9V}}

Option 1: Change the sign of the simulated value
\(-{\mathrm{Gamma}\left(\mathrm{mean}=\left|m_{\omega,d}\right|,\mathrm{var}=\phi\cdot\left|m_{\omega,d}\right|\right)}\)

\pandocbounded{\includegraphics[keepaspectratio]{https://storage.simpletex.cn/view/fHVFmGAaP93tlgXbXDGCpMkMSGOHul4Nt}}

Option 2: Shift the entire distribution to have a mean of
\(m_{\omega,d}\)

\pandocbounded{\includegraphics[keepaspectratio]{https://storage.simpletex.cn/view/f0VkRxsMxoKAet2PelpKdt72fLyrPlZKd}}

\begin{center}\rule{0.5\linewidth}{0.5pt}\end{center}

\subsubsection{Options to handle extreme outcomes caused by negative
incremental
losses:}\label{options-to-handle-extreme-outcomes-caused-by-negative-incremental-losses}

1.Identify and remove extreme iterations from results 2.Recalibrate the
model to correct the source of negative incremental losses (e.g.model
salvage and subrogation separately) 3.Limit incremental losses to a
minimum of zero

Non-Zero Sum of Residuals

If we desire an average residual of zero, then we can shift all the
residuals up (or down) by a constant so that the sum of the shifted
residualsis zero.Then, sample from the shiftedresiduals

\subsection{Using N-Year Weighted
Average}\label{using-n-year-weighted-average}

GLM Bootstrap Model:

·Only use the most recent \(N_{4}1\) diagonals to parameterize the model

\subsection{ODP Bootstrap Model:}\label{odp-bootstrap-model}

: Calculate \(N\) year LDFs and use to calculate residuals for the most
recent \(N+1\) diagonals

·Resample residuals for the whole triangle (because we need to calculate
cumulative losses)

·For each sample triangle,calculate new \(N\) year LDFs and use to
project future incremental losses

Missing Values

\subsection{GLM Bootstrap Model:}\label{glm-bootstrap-model-1}

·Missing data simply reduces the number of observations in the data.

\subsection{ODP Bootstrap Model:}\label{odp-bootstrap-model-1}

Estimate the missing value from surrounding values, OR ·Modify LDFs to
exclude missing values

\subsubsection{Outliers in the Original
Data}\label{outliers-in-the-original-data}

Can remove the value and treat like a missing value, OR Identify and
exclude outliers from the LDF calculation and residual calculations,but
re-sample the corresponding incremental loss for the sampled triangles.

\subsubsection{Heteroscedasticity}\label{heteroscedasticity}

\section{Stratified sampling:}\label{stratified-sampling}

Only sample residuals from within the heteroscedasticity group

Disadvantage -Some groups may have few residuals.This limits the
variability possible in the model.

\begin{center}\rule{0.5\linewidth}{0.5pt}\end{center}

\subsubsection{Hetero-adjustment factors based on standard deviation of
residuals:.}\label{hetero-adjustment-factors-based-on-standard-deviation-of-residuals.}

The hetero-adjustment factor adjusts the standardized Pearson residuals
to the average residual standard. deviation for the triangle as a whole.
With all the residuals at the same level of variability, the bootstrap
model can sample residuals from the entire triangle.

Hetero-adjustment factors based on different scale parameters

Instead of using standard deviations of residuals, use the ratio of the
square roots of the overall scale parameter and the scale parameter for
the groups based on age. Then, the hetero-adjustment factors are used
the same way as above during simulation.

\subsection{Heteroecthesious Data}\label{heteroecthesious-data}

\subsection{Partial first development period
data:}\label{partial-first-development-period-data}

·Reduce future incremental losses for the latest accident year to
correspond to the earned exposure.

\subsection{Partial last calendar period
data:}\label{partial-last-calendar-period-data}

a)Annualize exposures in the last diagonal so they're consistent with
the rest of the triangle b)Calculate the fitted triangle and residuals.
c)During the ODP bootstrap simulation, calculate and interpolate LDFs
from the fully annualized sample triangles. d)The last diagonal of the
sample triangle should be adjusted to de-annualize incrementals on the
latest diagonal. e)Project future values by multiplying the interpolated
LDFs with the new cumulative values. f)Reduce thefuture incremental
values for the latest accident year to remove future exposure (because
this is a partial first development period)

\subsection{Exposure Adjustment}\label{exposure-adjustment}

If there is rapid growth or a book is put in runoff, the exposure level
can change significantly year-to-year.

a)Divide all loss data by exposures for each accident year to get pure
premium. b)Run the model based on pure premium. c)At the end,multiply
results by the exposure level to get the total value.

Advantage In the GLM framework, with the exposure adjustment, we can use
fewer AY parameters

\section{Tail Factors}\label{tail-factors}

Instead of using a deterministic tail factor, we can assign a
distribution to the tail factor parameter. This makes the tail factor
stochastic within the bootstrap model.

\subsubsection{Parametric Bootstrap}\label{parametric-bootstrap}

If there is a lack of extreme residuals,we can fit a distribution to the
residuals.Instead of sampling with replacement directly from the
residuals, we sample residuals from the parameterized residual
distribution.. This will allow for more extreme residuals in the sampled
triangles

\begin{center}\rule{0.5\linewidth}{0.5pt}\end{center}

\subsection{Diagnostics}\label{diagnostics-1}

You should understand the different diagnostics used to assess the model
reasonableness. You should be able to read graphs and output and draw
conclusions about the model. Make sure you know what to expect from each
diagnostic.Definitelyreview the graphs and tables in the Shapland
appendices.

\subsection{Purpose of diagnostics:}\label{purpose-of-diagnostics}

·Test model assumptions ·Gauge how well the model fits the data ·Help
guide adjustments of model parameters

\subsubsection{Residual Graphs}\label{residual-graphs-1}

Residuals should be random around zero.You should look at the following
types of residual graphs:

Residuals vs.Calendar Year Residuals vs.~Development Period (look for
heteroscedasticity) Residuals ys. Accident Year Residuals vs.Fitted
Incremental Loss

\subsubsection{Normality Test}\label{normality-test}

This test compares the residuals to the normal distribution.Normality
isn't required, but Shapland does mention this as a diagnostic to
review. If residuals are close to normal, you should see:

·Normality plot with residuals in line with the diagonal line (normally
distributed) ·High \(\mathbb{R}^{2}\) value and p-value greater than
\(5\%\)

AIC and BIC can be used to compare different models to see which modelis
closest to Normal while penalizing for extra parameters (a lower value
is better):

\[\mathrm{AIC}=2\:p+n\Bigg[\ln\Bigg(\frac{2\pi\times\mathrm{RSS}}{n}\Bigg)+1\Bigg]\quad\mathrm{BIC}=n\times\ln\Bigg(\frac{\mathrm{RSS}}{n}\Bigg)+p\times\ln(n)\]

\subsection{Outliers}\label{outliers}

Use a box-whisker plot to help identify outliers. It's important to be
careful about removing outliers. If outliers represent scenarios that
can't be expected to ever happen again, then it may make sense to remove
them. But, ``outliers'' may simply be extreme, but realistic, values. If
this is the case, then the ``outlier'' should stay in the data.

\subsubsection{Parameter Adjustment}\label{parameter-adjustment}

The GLM Bootstrap model is more flexible, so we can look at the residual
graphs and adjust the parameters in the model by adding or removing
additional parameters.For instance,if there is a trend in the average
residuals by calendar year,then we might add a calendar year parameter
to the model.

\begin{center}\rule{0.5\linewidth}{0.5pt}\end{center}

\subsubsection{Model Results}\label{model-results}

You should see the following trends in the estimated unpaid model
results by accident year:

\subsubsection{Standard Error:}\label{standard-error}

·Standard error should increase from older to more recent years ·Total
standard error should be larger than the standard error for any
individual year

\subsubsection{Coefficient of Variation}\label{coefficient-of-variation}

·CoV should decrease from older to more recent years ·Total CoV should
be less than any individual year ·CoV could rise in the most recent
years due to: OParameter uncertainty in the most recent years could
overpower process uncertainty OThe model could be overestimating
uncertainty (might use the BF method instead

\subsection{Multiple Models}\label{multiple-models}

Model risk is the risk that the model doesn't reflect the ``true''
loss-generating process for future losses Weighing together multiple
reasonable models helps address model risk

\subsection{Correlation}\label{correlation}

\subsubsection{Location Mapping}\label{location-mapping}

For each iteration:

a)Sample residuals for segment 1 b)Track the location in the residual
triangle where each sampled residual was taken c)For all other segments,
sample the residuals from their residual triangles using the same
locations

\subsection{Advantages:}\label{advantages-2}

·The method is easily implemented .Doesn't require an estimated
correlation matrix

\subsection{Disadvantages:}\label{disadvantages-1}

·Requires all business segments to have the same size data triangles
with no missing data Correlation of original residuals is used, can't
test other correlation assumptions

\subsubsection{Re-sorting}\label{re-sorting}

Uses algorithms such as a copula to add correlation

\section{Advantages:}\label{advantages-3}

·Data triangles can be different shapes/sizes by segment

\begin{center}\rule{0.5\linewidth}{0.5pt}\end{center}

Can use different correlation assumptions

Different correlation assumptions may have other beneficial impacts on
the aggregate distribution oEx. Can use a copula with a heavy tail
distribution to strengthen the correlation between segments in the
tails,which is important for risk-based capital modeling

\section{Recipes for Calculation
Problems}\label{recipes-for-calculation-problems-6}

Over-Dispersed Poisson GLM Setup Expected Incremental Losses from GLM

Standardized Pearson Residuals

ODP Bootstrap Model

Negative Incremental Values

Heteroscedasticity Fix: Standard Deviation

Heteroscedasticity Fix: Scale Parameter Multiple Models: Same Random
Variables

Multiple Models: Independent Variables

\begin{center}\rule{0.5\linewidth}{0.5pt}\end{center}

\subsection{A Model for Reserving Workers Compensation High
Deductibles}\label{a-model-for-reserving-workers-compensation-high-deductibles}

\subsubsection{Overview}\label{overview-7}

The typical loss reserve methods are appropriate for ground-up losses.
Siewert focuses on different methods that can be used to estimate unpaid
loss reserves for different layers of losses, such as limited losses or
losses above a high deductible. One of the main challenges Siewert
addresses is how to calculate LDFs that are consistent between
unlimited, limited and excess layers.

\subsection{Excess Loss Reserving
Methods}\label{excess-loss-reserving-methods}

Loss Ratio Method

Siewert applies the loss ratio method by account to reflect differences
in account characteristics.There are two components to the expected
excess losses:

:Per Occurrence Charge - Expected losses above the per-occurrence
deductible. ·Aggregate Loss Charge -Expected losses in the deductible
layer (below the per-occurrence limit) that are above the aggregate
limit

\subsection{Advantages:}\label{advantages-4}

Useful for immature accident years when loss data is thin

Loss ratio estimates can be consistent with pricing

\subsection{Disadvantages:}\label{disadvantages-2}

Ignores actual loss experience, not that useful after several years
development

May not reflect account characteristics properly if development emerges
differently due to written exposures

\subsection{Implied Development}\label{implied-development}

The ultimate excess loss estimate is the difference between the
unlimited and limited ultimate loss estimates (calculated using the
unlimited and limited losses and LDFs)

\pandocbounded{\includegraphics[keepaspectratio]{https://storage.simpletex.cn/view/f114VsAoBm3qAWCC8VTXiBNdOB19gB6H8}}

The limited LDFs used to calculate \(UIt^{limited}\) need to reflect
indexed limits adjusted forinflation over time. This is because using a
constant deductible for all accident years implies increasing excess
losses.

\subsection{Advantages:}\label{advantages-5}

Can estimate excess loss for early maturities,even if no excess losses
have emerged yet ·Limited LDFs are more stable than excess LDFs used for
direct development

\begin{center}\rule{0.5\linewidth}{0.5pt}\end{center}

\subsubsection{Disadvantages}\label{disadvantages-3}

·Misplaced focus -we would like to explicitly recognize excess loss
development

\subsection{Direct Development}\label{direct-development}

Apply the XSLDF to excess loss to estimate ultimate excess loss. The
XSLDF should be consistent with the limited and unlimited LDFs

\section{Advantages:}\label{advantages-6}

·Focuses explicitly on excess loss development

\subsection{Disadvantages:}\label{disadvantages-4}

XSLDFs are often highly leveraged and volatile, especially for immature
periods ·Direct development doesn't produce an estimate if no excess
loss has emerged yet

\subsubsection{Credibility Weighting
(Bornhuetter-Ferquson}\label{credibility-weighting-bornhuetter-ferquson}

Excess ultimate loss estimate is a credibility weighting of the actual
experience and the expected losses (loss ratio method)

\section{Advantages:}\label{advantages-7}

·Can tie to thepricing estimate forimmature periods whenlosses haven't
emerged ·More stable estimate over time

\section{Disadvantages:}\label{disadvantages-5}

·Ignores actual loss experience to the extent of the complement of
credibility

\subsection{Development Model}\label{development-model}

The first way we can get LDFs for different limits is to calculate them
directly using full coverage loss experience. This way,we can directly
create loss triangles with losses limited at different limits.

\section{Indexed Limits}\label{indexed-limits}

Before calculating the limited LDFs from a limited loss triangle, we
need to index the limits for prior accident years.This is important so
that the proportion of limited-to-excess loss around the limit is
consistent from year to year.

To do this,Siewert fits an exponential curve to loss severity by
accident year to calculate the annual severity trend.Then,he uses the
selected severity trend to trend theloss limit backwards.Once wehave the
indexed limits,we can create a loss triangle with losses limited at the
appropriate indexed limits and calculate the limited LDFs.

See the examplebelow about how to calculateindexed limits:

\begin{center}\rule{0.5\linewidth}{0.5pt}\end{center}

\section{Example 1}\label{example-1}

Given:

·Annual loss severity trend is \(10\%\)

·Deductible for an account is \$250.00C

Calculate the indexed limits to use for a five-year limited loss
triangle.

\subsection{Solution}\label{solution}

First, set the limit for the most recent accident year to the stated
limit,\$250k. Then,trend the limits backward for each prior accident
year

\begin{longtable}[]{@{}ll@{}}
\toprule\noalign{}
Accident Year & Indexed Limit \\
\midrule\noalign{}
\endhead
\bottomrule\noalign{}
\endlastfoot
2012 & 170,753 \\
2013 & 187,828 \\
2014 & 206,612 \\
2015 & 227,272 \\
2016 & 250,000 \\
\end{longtable}

\textbackslash(=250,000 =
\textbackslash frac\{1\}\{1.10\}\textbackslash)

\subsubsection{Tail Factors}\label{tail-factors-1}

Siewert uses the inverse power curve to get consistent tail factors by
limit. First, he fits an inverse power curve to the unlimited LDFs and
select an appropriate stopping point beyond which there's no further
development (e.g.~40 years). He selects the stopping point so that the
inverse power curve tail factor is consistent with a tail factor based
on an extended loss triangle.

To get the tail factors for the other limits, he fits the inverse power
curve to the limited LDFs at each limit up to the stopping point and
compound the age-to-age factors from the fitted curves.

Relationships between unlimited/limited/excess LDFs and severity
relativities

There is a relationshipbetween limited/excess LDFs and unlimited LDFs
using loss severity relativities

R\_t\^{}L = \textbackslash begin\{cases\} Severity Limited to limit L at
age t \textbackslash\textbackslash{} Unlimited Severity at age t
\textbackslash end\{cases\}

Severity relativities should have the following relationships

:Severity relativity should decrease as age increases OThis is because
more losses are capped at the per-occurrence limit as age increases
:Severity relativity should be higher for a larger limit 0This is
because a higher limit means less of the loss is capped, so the
relativity is higher

You can see these relationships in the following graph of severity
relativities by age and limit

\begin{center}\rule{0.5\linewidth}{0.5pt}\end{center}

\pandocbounded{\includegraphics[keepaspectratio]{https://storage.simpletex.cn/view/f9uEStS4PxH7LTetwiTe5YvSzYwCb74Nt}}

Below are the key relationships:

\pandocbounded{\includegraphics[keepaspectratio]{https://storage.simpletex.cn/view/f02b7W9ElzbfGNE42pc6EaRavHcgpqMPT}}

The unlimited LDF is a weighted average of the limited and excess
LDFs.With this relationship, we can split the unlimited loss development
consistently between development below and excess the limit.

\pandocbounded{\includegraphics[keepaspectratio]{https://storage.simpletex.cn/view/fZ1el2LGmup6gZU2bL0pmBM2DOvA4lIyn}}

By using the severity relativities alone to adjust the LDF,we implicitly
assume no further claim count development.e

\subsection{Distribution Model}\label{distribution-model}

If we use the loss data directly,we'll find instances where limited
development sometimes is greater than the unlimited development. This
will happen if the limited severity relativity increases from one age to
the next (remember, the severity relativity should decrease as age
increases)

Instead of using the data directly,we can fit a separate severity
distribution to losses at each age (the severity distribution for losses
at 12 months will be different than the ultimate severity distribution).
Then,we can calculate the severity relativities at each age for any
limit we're interested in.

Siewert calculates the unlimited LDFs directly from the fitted severity
relativities.Then, using the fitted severity relativities and these
unlimited LDFs, we can then calculate consistent LDFs at any other
limit.

\subsection{Fitting the Model}\label{fitting-the-model}

Siewert uses a Weibull distribution and estimates the parameter by
minimizing the chi-square error between the actual and expected severity
relativities.Also,Siewert constrains the model so that the unlimited
expected severity is equal to the actual unlimited severity at that
development age

\subsubsection{Claim Count Development
Assumption}\label{claim-count-development-assumption}

In the development model, we started with selected unlimited LDFs based
on unlimited loss data. A key difference here is that we're calculating
the LDFs directly from the modeled severities.By using the severity
relativities alone to calculate the LDFs, we implicitly assume no
further claim count development. Because of this,an important assumption
is that there's no further claim count development

\begin{center}\rule{0.5\linewidth}{0.5pt}\end{center}

In Siewert (see pg. 232), he discusses how we can use the modeled
severities to calculate the LDFs, but specifies that if there is claim
count development for the earlier maturities,it needs tobe reflected.

\subsubsection{Partitioninq Loss Development above and below the
deductible}\label{partitioninq-loss-development-above-and-below-the-deductible}

Because the unlimited LDF is a credibility weighting of the limited and
excess LDFs, we can partition expected future development between
development above and below the deductible.

As age increases, a larger portion of the expected development will be
due to excess development above the limit.

\%Unpaid =1-LDE - LDF-1 R-,(LDF -1)+(1-R)(XSLDF -1)\_ R (LDF* -1) (1-
R5)-(XSLDF -1) Development belowDevelopment above the deductible the
deductible

Example 2

At 48 months development,below are the paid LDFs around the \(250k\)
deductible:

XSLDF Limited LDF Unlimited LDF 2.1521.9593.148 The 250k severity
relativity at 48 months is 0.838

Partition the expected future paid development above and below the 250k
deductible

\subsubsection{Solution}\label{solution-1}

\[
\begin{aligned}
96Unpaid& =1-\frac{1}{2.152}=53.5\% \\
&=\frac{.838\cdot\left(1.959-1\right)}{2.152}+\frac{\left(1-.838\right)\cdot\left(3.148-1\right)}{2.152}=37.3\%+16.2\%
\end{aligned}
\]

The expected development of \(53.5\%\) is made up of \(37.3\%\) of
development below the deductible and \(16.2\%\) of development above the
deductible.Put another way, \(69.7\%\) =.373/.535) of expected future
development is below the deductible and \(30.3\%\) is above the
deductible

\subsubsection{Aggregate Excess of Loss
Estimate}\label{aggregate-excess-of-loss-estimate}

An aggregate limit caps losses in the deductible layer that the insured
is responsible for.We can't calculate development factors for losses
excess the aggregate limit because the data needed to do this is thin
and not likely to be credible.

Instead, Siewert recommends using a collective risk model or Table M to
estimate liabilities under an aggregate limit.

\begin{center}\rule{0.5\linewidth}{0.5pt}\end{center}

\subsection{Collective Risk Model}\label{collective-risk-model}

Siewert uses the fitted Weibull model for the severity distribution and
a Poisson claim count distribution for frequency in the collective risk
model. The collective risk model provides expected aggregate excess loss
in the deductible layer at each age and at ultimate (using the different
fitted Weibull models for each age) Then, the expected excess losses are
used to calculate the aggregate excess of loss LDFs.

Because losses excess the aggregate are so volatile,Siewert recommends
using the BF method

These are the key relationships you should notice:

. Aggregate excess loss development drops off faster for smaller
per-occurrence deductible limit

OBecause later development is more likely for larger claims that already
are over the deductible (so they don't impact the aggregate limit)

Higher aggregate limits have more leveraged LDFs OBecause there are
fewer losses excess the aggregate limit when the aggregate limit is
high, especially at earlier periods

\subsection{Table M Approach}\label{table-m-approach}

An aggregate limit is similar to a maximum premium limit used in
retrospective rating plans with Table M. This is covered in Exam
8,soIwouldnt expect a significant question on the exam about this
approach,but Siewert does show an example calculation in the appendix

An advantage of this approach compared to the collective risk model
approach is that it can be more practical.

\subsubsection{Service Revenue}\label{service-revenue}

In a high deductible program, the insured is responsible for the
deductible losses that are below the aggregate limit, but the insurer
still services those claims

·Service Revenue -Reflects the costs the insurer charges the insured for
servicing those claims. Service Revenue Asset -The expected ultimate
service revenue minus the recoveries for service revenue to-date.

\subsection{Recipes for Calculation
Problems}\label{recipes-for-calculation-problems-7}

Loss Ratio Method

.Implied Development Method Direct Development Method Credibility
Weighted Method Unlimited/Limited/Excess LDF Relationships
.Distributional Model ·Aggregate Excess of Loss Ultimate ·Aggregate
Excess of Loss Ultimate: Table M ·Service Revenue Asset

\begin{center}\rule{0.5\linewidth}{0.5pt}\end{center}

Claims Development by Layer

\subsubsection{Overview}\label{overview-8}

The goal of this paper is to show the relationship between loss trend, a
claim size (severity) model and cumulative loss development factors.
With this relationship, we can adjust CDFs from a base'' layer to
another layer and from one cost level to another.

\subsubsection{Trend}\label{trend}

The loss trend factors we use are for the ground-up unlimited loss layer
and apply to cumulative loss, not incremental loss. We create a trend
table of the cost level indices. This way, we can reflect trends that
act in different directions such as an accident year trend or a calendar
year trend

\subsubsection{Claim Size (Severity)
Model}\label{claim-size-severity-model}

The loss severity distribution is used to calculate limited expected
values (LEVs) at different limits, which are used to adjust CDFs from
one layer to another.

We start with a severity distribution at the current cost level for each
development period with different parameters for each age. This is
similar to how Siewert uses a different Weibull severity distribution
for each age.Then, using the trend, we can find the appropriate
parameters (unlimited expected claim size) for the severity
distributions at the prior accident year cost levels by trending
backwards.

\subsection{Loss Development}\label{loss-development}

Loss development factors are calculated as volume-weighted LDFs and CDFs
using the chain ladder method

\subsection{Relationship Between
Layers}\label{relationship-between-layers}

Remember, the goal of this paper is to use the relationship between
trend, a claim size model and loss development to calculate appropriate
CDFsfor any other layer and cost level

For instance, we might have a triangle of ground-up loss data limited to
\$1,00o,000 and a loss triangle from a book of business with a \$50o,000
high deductible. We can use the relationship between these three
components with the method below to calculate the appropriate CDFs that
we can apply to the latest diagonal to estimate ultimate excess loss for
the book of business.

\subsection{Getting CDFs ata
basiclimit}\label{getting-cdfs-ata-basiclimit}

Normally, we calculate the LDFs directly from an unadjusted loss
triangle and use them to develop losses to ultimate for all accident
years. BUT, this is only appropriate if

·The loss triangle is ground-up unlimited AND ·Trend onlyacts in the
accident-year direction

\begin{center}\rule{0.5\linewidth}{0.5pt}\end{center}

Often this isn't the case,so we first adjust the loss triangle to get
data to the same cost level and to a basic per-occurrence limit where we
believe the data is credible for estimating LDFs.

To do this, we first calculate the unlimited expected claim size at each
age trended backwards. Then, we can calculate the LEVs to adjust the
loss data.We need the LEVs at the limit of the data and accident year
cost level (where the loss data is now). We also need the LEVs at the
basic limit and the most recent accident year cost level.

With the LEVs, we can adjust the loss data so that the whole triangle is
now at the basic limit and current accident year cost level. There are
two adjustments going on at the same time:

LossLimit -Adjustfrom theloss limit of the original data to
thebasiclimit ·Cost Level- Adjust from the separate accident year cost
levels to a common cost level

\[\boxed{C_{AY,k}^{\prime}=C_{AY,k}\cdot\frac{\mathrm{LEV}\left[B\mid\theta_{n,k}\right]}{\mathrm{LEV}\left[L\mid\theta_{AY,k}\right]}}\quad Adjusted\:Loss_{AY,k}=Loss_{AY,k}\cdot\frac{\mathrm{LEV}\left[B_{Latot,AY,k}\right]}{\mathrm{LEV}\left[L_{AY,k}\right]}\]

Then, we calculate the LDFs and CDFs from the adjusted loss triangle.
Now we have a development pattern that's at the basic limit and a common
cost level.

\subsubsection{Getting CDFs at any other layer and accident year cost
leve}\label{getting-cdfs-at-any-other-layer-and-accident-year-cost-leve}

Once we've got the development pattern at a basic limit and common cost
level, we can use the LEVs from the claim size model to adjust the
development pattern to any other limit and cost level. Most importantly,
we can get the CDFs that we can apply to the most recent loss diagonal
to develop losses.

The formula below uses the relationship between trend, claim size model
and CDFs to calculate the adjusted CDFs:

\pandocbounded{\includegraphics[keepaspectratio]{https://storage.simpletex.cn/view/fYDlQGUqK7tdT154FCRn5rZmgdPWlaR5d}}

Again, there are two adjustments going on in this formula:

·Adjust CDFs from the basic limit to layer X ·Adjust CDFs from the
common cost level to separate accident year cost levels along the
diagonal

Now, we can use these adjusted CDFs with a loss reserving method to
estimate loss reserves for a loss triangle at the new layer, layer X.

\subsection{Simplified Model}\label{simplified-model}

A disadvantage of this method is that it requires a claim size model at
each development age.We might only know the severity distribution and
the expected unlimited severity atultimate

In this case, we can use a simplified model (below) to adjust the basic
limit CDFs to the new layer for the most recent diagonal:

\begin{center}\rule{0.5\linewidth}{0.5pt}\end{center}

\(F_{A,k}^X = \frac{LEV[X|\theta_{A,k}]}{LEV[B|\theta_{A,k}]}R_k\)

Wheei
\(R_{\varepsilon}=\frac{\mathrm{LEV}\left[X\mid\theta_{AY,k}\right]}{\mathrm{LEV}\left[B\mid\theta_{AY,k}\right]}\)

In this model, \(R_{j}\) is the ratio between the limited severities at
layer X and layer B. This is similar to the severity relativity in the
Siewert paper, but remember Siewert always looks at the severity
relativity of limited-to-unlimited loss. Here, we're looking at the
severity relativity between limited losses at two different limits.

Besides using a ratio for the denominator,note that this simplified
variation doesn't do the adjustment from one cost level to another.This
method makes the assumption that the differences in cost level
areimmaterial to the ratios of the LEVs by layer.

If the new layer, X,is lower than the basic limit, then \(R_{j}<1\) and
Sahasrabuddhe points out that should see the following properties:

\subsection{·R,decreases as age
increases.}\label{rdecreases-as-age-increases.}

OAt earlier maturities,most of the development will be in the lower
layer below X, so \(R_{,}\) will be close to 1. OAt later development
periods,there will be more development in the layer between limits X and
B. This means \(R_{j}\) will decrease for later development periods.

\subsection{\texorpdfstring{\(R_{j}>\mathbf{U}=\lim R_{j}\) at ultimate,
calculated as the ratio of LEVs at ultimate between layer X and
B}{R\_\{j\}\textgreater\textbackslash mathbf\{U\}=\textbackslash lim R\_\{j\} at ultimate, calculated as the ratio of LEVs at ultimate between layer X and B}}\label{r_jmathbfulim-r_j-at-ultimate-calculated-as-the-ratio-of-levs-at-ultimate-between-layer-x-and-b}

oThis follows from the bullet above. \(R_{j}\) monotonically decreases
as maturity increases until losses are at ultimate and the ratio
\(R_{j}\) is at its minimum, U OAlso,Sahasrabuddhe points out that U can
be calculated as

\[U=R_{j}\cdot\frac{CDF^{x}}{CDF^{^B}}\]

oIn this formula, the ratio of CDFs develops both the numerator and
denominator of \(R_{j}\) to ultimate, giving the ultimate ratio \(U.\)

\subsection{\texorpdfstring{·If the base layer CDFs are ground-up
unlimited,then the upper limit of \(R_{j}\) is
\(\max(Rj)=\mathbf{U}\times\mathbf{CDF}\)}{·If the base layer CDFs are ground-up unlimited,then the upper limit of R\_\{j\} is \textbackslash max(Rj)=\textbackslash mathbf\{U\}\textbackslash times\textbackslash mathbf\{CDF\}}}\label{if-the-base-layer-cdfs-are-ground-up-unlimitedthen-the-upper-limit-of-r_j-is-maxrjmathbfutimesmathbfcdf}

OIn this case,the Basic Limit,B,is Ground-Up Unlimited. 0For any
maturity jbefore ultimate,the maximum of what the ratio \(R_{j}\) can be
is when losses limited at X are fully developed as of time j and all
future development after \(j\) is in the layer excess limit X. This
would mean \(\mathrm{CDF}_{\mathrm{Layer~X}}=1.0\)C OIn this case, the
ratio at ultimate can be calculated as the following by plugging into
the formula from above:

\[U=R_{_{j,\max}}\cdot\frac{1.00}{CDF^B}\]

ORearranging this and since layer \(B\) is ground-up unlimited, we have:

\[R_{_{j,\max}}=U\cdot CDF\]

\begin{center}\rule{0.5\linewidth}{0.5pt}\end{center}

\subsection{Assumptions}\label{assumptions}

·Actuary must select the basic limit ·Requires a claim size model (but
with the simplified model we only need it at ultimate) ·Loss
trianglemust be adjusted to basic limit and a common cost level

\subsubsection{More difficult
assumptions}\label{more-difficult-assumptions}

Claim size models at earliermaturities aren't usually available OThe
simplified version only uses the ratio of LEVs at earlier maturities
which could be easier to estimate ·Must calculate trend indices
appropriate for cumulative losses,which may be difficult because: oTrend
typically impacts incremental losses OTrend on reported losses and the
timing of trend on case reserves is difficult to get at

\subsection{Interpreting Results}\label{interpreting-results}

We can see how this approach impacts the development pattern if we take
a limited loss triangle at limit L and compare the CDFs calculated
directly from the triangle to the CDFs calculated from the method above
(first adjusting to a basic limit and common cost level, then adjusting
back to limit L and the cost level for the latest diagonal).

Examples 1 and 2 in the paper appendix shows this comparison. Both
examples use the same data but Example 2 uses a severity model with
larger claim sizes. Example 2 shows greater differences between the
calculated CDFs with the Sahasrabuddhe method and the unadjusted CDFs
(from the original loss triangle).

\subsubsection{Key relationships}\label{key-relationships}

Differences are greater for larger expected unlimited claim size which
increases the expected loss in the layer between the basic limit and the
new limit. Differences are greater where trend and/or loss development
act over longer time periods (longtailed lines) or when the loss trend
is higher.

The differences are smaller for short-tail lines, low trend rates and
limits above the working layer (limits significantly higher than the
average severity)

\subsection{Recipes for Calculation
Problems}\label{recipes-for-calculation-problems-8}

·Determine LDFs at Basic Limits Calculate Layer LDFs from Basic Limit
LDFs ●Calculate Layer LDFs from Basic Limit LDFs:Simplified Model

\begin{center}\rule{0.5\linewidth}{0.5pt}\end{center}

\subsection{Estimating the Premium Asset on Retrospectively Rated
Policies.}\label{estimating-the-premium-asset-on-retrospectively-rated-policies.}

\subsubsection{Overview}\label{overview-9}

Teng and Perkins is a calculation-heavy paper focused on how to estimate
the premium asset for retrospectively rated policies. The key to this
method is the relationship between loss development and premium
development (the PDLD ratios).

Make sure to thoroughly review the exhibits at the end of the paper.
They show all of the steps in the calculation of the final premium asset
from the original premium and loss triangles.

\subsubsection{Retro Rated Policies}\label{retro-rated-policies}

Reasons why Retro Rated Policies are Popular

·Premium is returned to the insured for good loss experience Cash flow
advantage - The insured can pay premium as losses are reported or paid,
instead of upfront ●More risk is shifted to the insured -Final premium
varies with loss experience (good for insurer)

\subsection{Retro Premium Formula}\label{retro-premium-formula}

The premium for a retro policy at the \(n^\mathrm{th}\) adjustment is
givenby the following formula:

\[Prem_{n}=\left(\begin{matrix}Basic&Prem+Capped&Loss_{n}\cdot Loss&Conversion&Factor\end{matrix}\right)\cdot TaxMultiplier\]

Basic Premium (BP) - Covers company expenses, the insurance charge for
the max/min premium, and the excess loss charge for a per-accident loss
limit Capped Losses \(\mathrm{'CL}_{\mathrm{n.}}\) )-Losses that
contribute to additional premium 0Losses are capped to a min/max
corresponding with a min/max premium for the policy and individual
claims may be capped as well. Loss Conversion Factor (LCF) Covers loss
adjustment expenses ·Tax Multiplier (TM)-Covers state premium taxes and
other state assessments

\subsection{Fitzgibbon Method}\label{fitzgibbon-method}

The idea of the Fitzgibbon method is that the retro formula is
essentially a linear equation and instead of matching the individual
policy retro rating parameters to the individual loss experience, we can
run a linear regression to get the average retro rating parameters for
the linear equation.

\begin{center}\rule{0.5\linewidth}{0.5pt}\end{center}

The Fitzgibbon method, the way Feldblum shows it,uses this approach

\pandocbounded{\includegraphics[keepaspectratio]{https://storage.simpletex.cn/view/fQh49Yh36TlMurZcySLUaeKGT0XnhHnc2}}
\(B=\) Premium Responsiveness

Fitzgibbon estimates the parameters for the model, A and B, using a
linear regression of retrospective adjustments for mature policies
against the standard loss ratios for those policies

\subsubsection{Reasons why the slope ( doesnt equal
1}\label{reasons-why-the-slope-doesnt-equal-1}

Losses over theloss limit or beyond the max premium don't increase the
retrospective premium

·The minimum premium is greater than the basic premium for some plans
·The loss conversion factor (LCF) and tax multiplier (TM) change the
slope

\subsubsection{Disadvantages of the Fitzqibbon
method}\label{disadvantages-of-the-fitzqibbon-method}

Instead of looking at future premium adjustments for expected future
losses (which is what the PDLD method does),Fitzgibbon calculates the
ultimate premium estimate based off of cumulative loss-to-date.This is a
problem if the premium-to-date is less than what's expected based on
current losses-to-date. Because of the loss limits and max/min
retrospective premiums,we expect premium sensitivity (the slope, \(B)\)
to decrease formore mature losses and higher overall loss ratios,but the
Fitzgibbor method has a constant slope

\subsection{Example 1}\label{example-1-1}

A=1,000 Losses at 36 months are 1,000 ·B=1.2 ·Expected Ultimate Loss is
5,000

Calculate the future premium based on the following values of
retrospective premium as of 36 months

· \(\mathrm{Prem}_{36}=2,200\) (Expected retro premium for
\(\mathrm{Loss}_{36}=1,000]\) . \(\mathrm{Prem}_{36}=1,500\)

\section{Solution}\label{solution-2}

\[EstUltPrem=1,000+1.2\times5,000=7,000\]

\begin{longtable}[]{@{}
  >{\raggedright\arraybackslash}p{(\linewidth - 6\tabcolsep) * \real{0.2754}}
  >{\raggedright\arraybackslash}p{(\linewidth - 6\tabcolsep) * \real{0.2899}}
  >{\raggedright\arraybackslash}p{(\linewidth - 6\tabcolsep) * \real{0.2319}}
  >{\raggedright\arraybackslash}p{(\linewidth - 6\tabcolsep) * \real{0.2029}}@{}}
\toprule\noalign{}
\begin{minipage}[b]{\linewidth}\raggedright
Retro Premium
\end{minipage} & \begin{minipage}[b]{\linewidth}\raggedright
Premium at 36 Months
\end{minipage} & \begin{minipage}[b]{\linewidth}\raggedright
Est. Ult Premium
\end{minipage} & \begin{minipage}[b]{\linewidth}\raggedright
Future Premium
\end{minipage} \\
\midrule\noalign{}
\endhead
\bottomrule\noalign{}
\endlastfoot
Expected & 2,200 & 7,000 & 4,800 \\
Lower-than-Expected & 1,500 & 7,000 & 5,500 \\
\end{longtable}

Using the Fitzgibbon method, the estimated ultimate premium only depends
on the overall ultimate loss estimate. If the premium responsiveness is
less-than-expected for losses-to-date (the second situation), the
estimated future premium will increase so that the overall estimated
ultimate premium stays the same

\begin{center}\rule{0.5\linewidth}{0.5pt}\end{center}

\subsection{Teng and Perkins PDLD
Method}\label{teng-and-perkins-pdld-method}

The PDLD method is a bit like the BF method. It focuses on future
premiums based on future expected losses, independent of prior
experience

\subsection{Advantages of the PDLD
Method}\label{advantages-of-the-pdld-method}

It's directly modeled on the retrospective rating formula, so it's
easily explainable ·Its focus on premium sensitivity is similar to how
loss sensitive contracts are handled under Risk Based Capital ●May be
useful when changes to retro rating parameters distorts indications from
other methods

\subsubsection{Reasons to use the PDLD Method Instead of the Chain
Ladder
Method}\label{reasons-to-use-the-pdld-method-instead-of-the-chain-ladder-method}

·Timeliness -We can estimate ultimate loss sooner than we can estimate
retrospective premiums. Also, we can update the premium asset estimate
quarterly with the PDLD method. Retrospective premiums depend on
incurred losses

\subsubsection{Teng and Perkins PDLD
Assumptions}\label{teng-and-perkins-pdld-assumptions}

Premium responsiveness for future adjustments is independent of premium
responsiveness of past adjustments.

The slope of the line segment (premium responsiveness for an adjustment
period) is independent of the beginning loss ratio and or the beginning
retro premium ratio.

\subsubsection{Reasons It's Difficult to Use Retro Rating Parameters
from
Pricinq}\label{reasons-its-difficult-to-use-retro-rating-parameters-from-pricinq}

Different insureds have policies with different retro rating parameters
(e.g.~a policy for a small insured may have a lower max premium than a
large insured Premium taxes differ between states Basic premium may vary
over time

\subsection{General Rule for Premium
Responsiveness}\label{general-rule-for-premium-responsiveness}

·Premium responsiveness decreases as a book of business matures. (larger
\(0\%\) of losses are excluded due to loss limits and max premium
Premium responsiveness decreases for higher overall loss ratios.
(policies are more likely to have hit max premium)

\begin{center}\rule{0.5\linewidth}{0.5pt}\end{center}

\subsection{Premium Development to Loss Development (PDLD)
Ratios}\label{premium-development-to-loss-development-pdld-ratios}

Method 1-PDLD Ratios based on the Retro Formula

Advantages of the RetroFormula Method Disadvantages of the Retro Formula
Method ·Responds to changes in the retro rating·Must select retro rating
parameters, which can parameters for sold policies be difficult because
parameters will vary between policies sold (Teng recommends using the
average of the sold parameters)

If the retro rating parameters change significantly over time, PDLD
ratios from the retro formula method should be given more weight than
PDLD ratios based on historical data.The retro formula can directly
reflect those changes.

\subsubsection{Method 2-PDLD Ratios based on Historical
Data}\label{method-2-pdld-ratios-based-on-historical-data}

For this method, we use booked premium and reported loss development to
calculate the PDLD ratios. Data should be partitioned to homogenous
groups based on account size and type of rating plan. Policies are
grouped by effective date, so data is based on policy effective quarter
(not accident year).

Premium for a retro-rated policy is evaluated at each``retro adjustment
period.'' The first retro adjustment typically uses losses at 18 months
development. Then, retro adjustments occur every 12 months after that.
Premiums are assumed to be booked 3 to 9 months after the losses used to
calculate the premium for the retro adjustment.

\subsubsection{Reviewing PDLD Ratios}\label{reviewing-pdld-ratios}

·Look for trends in the historical PDLD ratios by policy quarter.An
upward trend could indicate

0More ``liberal''\,'' rating parameters that increase the capped losses
and therefore increase premium (e.g.~higher maximum premium or
per-accident limit). OBetter loss experience, causing more of the losses
to be within the capped limits.

Historical PDLD ratios may be volatile after the first retro adjustment
because incremental premium development may reflect incremental loss
development on only a small number of policies.

It's possible to see negative incremental historical PDLD ratios when
loss development is positive.

OThere could be upward development on a claim that already passed the
per-accident limit (no change in premium) while other losses within the
limits fall (causing premium to fall).

Compare the average of many quarters of the historical PDLD ratios
toPDLD ratios from the retroformula to select aPDLD ratio.

PDLD ratios for the first adjustment are typically greater than 1
because of theBasicPremium and because few losses are capped.

\begin{center}\rule{0.5\linewidth}{0.5pt}\end{center}

PDLD ratios should generally fall for more mature retro adjustments
because more losses will be capped.

\subsubsection{Reasons Historical PDLD Ratios may differ from the Retro
Formula PDLD
Ratios}\label{reasons-historical-pdld-ratios-may-differ-from-the-retro-formula-pdld-ratios}

Worse (or better) than expected loss experience can cause more (or less)
losses to be capped resulting in historical PDLD ratios that are smaller
(or larger) than the retro formula PDLD ratios. ·Average retro rating
parameters may change over time.

\subsubsection{Loss Capping Ratios}\label{loss-capping-ratios}

The loss capping ratio typically decreases as losses mature since more
losses are outside the loss limits (aggregate limits for min/max premium
and per-accident limits).

For the exam, it's important to know the difference between incremental
and cumulative loss capping ratios. The retro formula uses
incrementa/loss capping ratios so if the problem gives you cumulative
loss capping ratios,you need to adjust them to incremental ones
first.Ifit'sunclear,make sure to state your assumption

Incremental Loss Capping Ratios

Cumulative Loss Capping Ratios

\[\frac{CL_n}{L_n}=\frac{Capped\:Loss_n}{Loss_n}\frac{\Delta CL_n}{\Delta L_n}=\frac{Capped\:Loss_n-Capped\:Loss_{n-1}}{Loss_n-Loss_{n-1}}\]

If you're given cumulative loss capping ratios, you can convert to
incremental loss capping ratios. The formula is shown in section 5,which
is off the syllabus,but also shows up in exhibit 5 in the appendices.I
think it still could be testable and a few past CAS problems are
conflicting regarding incremental vs.~cumulative loss capping ratios, so
it's a good idea to be prepared for it.

\pandocbounded{\includegraphics[keepaspectratio]{https://storage.simpletex.cn/view/fQViXlBUPGO6RrzHuoTvLSu2GWw0tVqZp}}

\subsection{Cumulative Premium Development to Loss Development (CPDLD)
Ratios}\label{cumulative-premium-development-to-loss-development-cpdld-ratios}

Remember that CPDLDratios relate toPDLD ratios the same way that
CDFsrelate toLDFs for the chain ladder method

The goal of the CPDLD ratios is to estimate the total expected future
premium development for a policy effective year by multiplying the CPDLD
by the expected future loss.

\subsubsection{Premium Asset}\label{premium-asset}

It's important to remember that the premium asset is calculated as the
difference between ultimate premium and current bookedpremium,NOT
premium booked at the prior adjustment.This is because there willbe some
differences between the booked premium at the evaluation date of the
analysis and the premium booked at the prior adjustment period.

\begin{center}\rule{0.5\linewidth}{0.5pt}\end{center}

We should also consider the collectability of the premium asset (similar
to recovery risk in Marshall) due to the possibility that the insured
won't pay future premium development.

\subsection{Enhanced PDLD Method - PDLD with a
constant}\label{enhanced-pdld-method---pdld-with-a-constant}

\(\mathrm{CPDLD}_{1}\) in the Teng and Perkins PDLD method includes both
the impact of losses on the retro premium and the basic premium. One of
the advantages of the Fitzgibbon method is that the basic premium
portion was broken out and reflected as a fixed amount.

Feldblum proposes an enhancement to the PDLD method by removing the
basic premium portion from the \(\mathrm{CPDLD}_{1}\) This way, the
adjusted CPDL\(D_{1}\) only reflects future premium development due to
future loss development and the basic premium is added afterwards as a
fixed amount.

\subsection{Recipes for Calculation
Problems}\label{recipes-for-calculation-problems-9}

PDLD Ratios - Retro Formula PDLD Ratios -Empirical ·Premium Asset
Cumulative to Incremental Loss Capping Ratios Fitzgibbon Method Premium
Asset -PDLD Method Enhancement

\begin{center}\rule{0.5\linewidth}{0.5pt}\end{center}

\subsection{Stochastic Loss Reserving Using Bayesian MCMC
Models}\label{stochastic-loss-reserving-using-bayesian-mcmc-models}

\subsubsection{Overview}\label{overview-10}

The main goal of this paper is to retrospectively test out different
Bayesian MCMC models and see which models perform best.

For each model (e.g.~Mack Model), Meyers takes 200 historical loss
triangles that have already developed to ultimate (or rather, to the end
of the 10-year development window).For each loss triangle,Meyers uses
the model to calculate a predicted distribution of ultimate losses.
Then, he looks at the percentile of where the actua/ultimate loss falls
within the predicted distribution.This is done for each of the 200
historical loss triangles.

Meyers tries to validate the different models by seeing if the
percentiles of the actual observations on the model distributions are
consistent with those expected.The model is validated if the observed
percentiles appear to be uniformly distributed.

\subsection{Tests to Validate Models}\label{tests-to-validate-models}

We can test the validity of a model by seeing if the percentiles of the
actual outcomes within the predicted distributions are uniform for a
large number of datasets (Meyers uses 200). To do this, use a
combination of the following three tests to see if the percentiles are
reasonably uniform

\subsection{Histogram and p-p Plot of Outcome
Percentiles.}\label{histogram-and-p-p-plot-of-outcome-percentiles.}

If the model validates and the percentiles are uniformly distributed,we
should see:

·Histogram -Bars of equal height

· \(P-P\) Plot - Sorted predicted percentiles (percentile of actual
outcome within the predicted distribution) should follow the expected
percentiles along a \(45^{\circ}\) line

\subsubsection{Kolmogoroy-Smirnoy Test}\label{kolmogoroy-smirnoy-test}

The K-S test compares the maximum distance between the predicted
percentiles and the expected percentiles (the K-S statistic) to a
threshold. If the K-S statistic is greater than the threshold, then the
test fails and we reject the null hypothesis that the percentiles are
uniformly distributed.

Visually, the K-S bands (indicating the threshold) can be plotted on a
\(p-p\) plot and if any of the points fall outside the bands,then the
test fails.

\subsection{Non-Bayesian Model
Results}\label{non-bayesian-model-results}

\subsection{Mack Model (Incurred
Losses)}\label{mack-model-incurred-losses}

Histogram shows more high/low percentiles than expected . \(p-p\) plot
shows a slanted ``S'' curve and the K-S test fails at the \(5\%\) level
for allines combined

\begin{center}\rule{0.5\linewidth}{0.5pt}\end{center}

Conclusion: The Mack model is too light in the tails (it underestimates
the variability of ultimate loss estimates)

\subsubsection{Mack Model (Paid Losses}\label{mack-model-paid-losses}

·Histogram and \(p^{-}p\) plot shows more low percentiles than expected
·K-S test fails at the 596 level for all lines combined

Conclusion:The Mack model on paid losses is biased high(expected loss
estimates are too high)

\subsubsection{Bootstrap ODP Model (Paid
Losses}\label{bootstrap-odp-model-paid-losses}

·Histogram and \(P-p\) plot show more low percentiles than expected ·K-S
test fails at the 596 level for all lines combined

Conclusion: The ODP Bootstrap model is biased high (expected loss
estimates are too high)

\subsection{Bayesian Incurred Loss
Models}\label{bayesian-incurred-loss-models}

Ways to improve the recognition of variability in the predictive
distributior

·The Mack model multiplies LDFs by a fixedlevel parameter, the
cumulative loss on the diagonal OWe can increase model variability by
treating the level of the accident year as random The Mack model assumes
independence between accident years OIncrease model variability by
allowing correlation between accident years (CCL model)

Leveled Chain Ladder (LCL)

\subsubsection{Key improvement over Mack LCL uses random level
parameters for each accident year (the Mack model uses fixed
losses-to-date)}\label{key-improvement-over-mack-lcl-uses-random-level-parameters-for-each-accident-year-the-mack-model-uses-fixed-losses-to-date}

\[\begin{aligned}&\mu_{\omega,d}=\alpha_{\omega}+\beta_{d}\\&Loss_{\omega,d}^{sim}\sim\mathrm{lognormal}\left(\mu_{\omega,d},\sigma_{d}\right)\end{aligned}\]

Correlated Chain Ladder (CCL)

\subsubsection{Key improvement over LCL: CCL allows for correlation
between accident
years.}\label{key-improvement-over-lcl-ccl-allows-for-correlation-between-accident-years.}

\[
\begin{aligned}
&\mu_{1,d}=\alpha_{1}+\beta_{d} \\
&\mu_{\omega,d}=\alpha_{\omega}+\beta_{d}+\rho\Big(\ln\Big(C_{\omega-1,d}\Big)-\mu_{\omega-1,d}\Big)\quad\mathrm{for}\:w>1 \\
&Loss_{\omega,d}^{sin}\sim\mathrm{lognormal}(\mu_{\omega,d},\sigma_{d})
\end{aligned}
\]

For both the CCL and LCL models, we can incorporate expert opinion by
setting more restrictive prior distributions for \(\alpha_{w}\) and
logelr , similarly to how expert opinion is incorporated in Verrall.

\begin{center}\rule{0.5\linewidth}{0.5pt}\end{center}

Results

BothLCL and CCL models produce higher standard deviations of predicted
outputs thanMack

\subsubsection{LCL Model}\label{lcl-model}

Histogram and \(p-p\) plot shows more high/low percentiles than expected
·K-S test fails at the 596 level for all lines combined ·Conclusion: The
LCL model is too light in the tails (but is better than Mack)

\subsubsection{CCL Model}\label{ccl-model}

.Correlation parameter, \(\rho\) ,is generally positive, resulting in
higher prediction variance than LCL Histogram and \(p-p\) plot shows
slightly more high/low percentiles than expected ·K-S test passes at the
\(5\%\) level for all lines combined Conclusion: The CCL model is
validated (but has mildly thin tails)

\subsubsection{Bayesian Paid Loss
Models}\label{bayesian-paid-loss-models}

Correlated Incremental Trend (CIT)

\subsubsection{Key improvement:}\label{key-improvement}

CIT includes a calendar-year trend for incremental paid losses CIT
allows for correlationbetween accident vears (similar to CCL)

\[
\begin{aligned}
&\mu_{\omega,d}=\alpha_{\omega}+\beta_{d}+\tau(w+d-1) \\
&Z_{_{w,d}}\sim\mathrm{lognormal}(\mu_{_{w,d}},\sigma_{_d}) \\
&IncLoss_{1,d}^{sim}\sim\mathrm{normal}\big(Z_{1,d},\delta\big) \\
&IncLoss_{\omega,d}^{sim}\sim\mathrm{normal}\Big(Z_{\omega,d}+\rho\cdot\Big(IncLoss_{\omega-1,d}^{sim}-Z_{\omega-1,d}\Big)\cdot e^{\tau},\delta\Big)& \mathrm{for}\:w>1
\end{aligned}
\]

\subsubsection{Differences from the CCL
model:.}\label{differences-from-the-ccl-model.}

· \(\sigma_{d}\) oCCL model is for cumulative losses, so \(\sigma_{d}\)
decreases for older development periods. oCIT model is for incremental
losses, so \(\sigma_{d}\) increases.

Autocorrelation oCCL:Autocorrelation is between the log of cumulative
losses oCIT: Autocorrelation is in the process variance step, not in
calculation of log mean, \(\mu_{w,d}\) because of the potential for
negative incremental losses. .Trend

\begin{center}\rule{0.5\linewidth}{0.5pt}\end{center}

oCCL:No trend oCIT: Trend factor is included in the calculation of the
log mean, \(\mu_{w,d}\) , and also included in the autocorrelation piece

\subsubsection{Changing Settlement Rate
(CsR}\label{changing-settlement-rate-csr}

\subsection{Key improvement:}\label{key-improvement-1}

CSRparameter reflects changes to the claims settlement rate (faster
payment pattern)

·Note: There's no autocorrelation or trend terms

\[
\begin{aligned}
&\mu_{\omega,d}=\alpha_{\omega}+\beta_{d}\cdot\left(1-\gamma\right)^{\omega-1} \\
&Loss_{_{w,d}}^{sim}\sim\mathrm{lognormal}(\mu_{_{w,d}},\sigma_{_d})
\end{aligned}
\]

\subsubsection{Results}\label{results}

\subsubsection{CCL Model}\label{ccl-model-1}

·Histogram and \(p-p\) plot shows more low percentiles than expected
·K-S test fails at the 596 level for allines combined ·Conclusion: The
CCL model is biased high on paid losses (expected loss estimates are too
high)

\subsubsection{CIT Model}\label{cit-model}

.Correlation parameter, \(\rho\) ,is generally close to zero :Trend
parameter, \(\tau\) , is predominantly negative, mostly offset by
increases to \(\alpha_{\mathrm{u}}\) and \(\beta_{d}\) ·Histogram and
\(p-p\) plot shows more low percentiles than expected ·K-S test fails at
the 596 level for all lines combined ·Conclusion: The CIT model is
biased high on paid losses (expected loss estimates are too high)

LIT Model (same as CIT model, but correlation parameter set to zero)

Histogram and \(p-p\) plot shows more low percentiles than expected ·K-S
test fails at the 596 level for all lines combined ·Conclusion: The LIT
model is biased high on paid losses (expected loss estimates are too
high)

\subsubsection{CSR Model}\label{csr-model}

Settlement rate parameter, Y , is generally positive, reflecting a
speedup in the payment pattern Histogram and \(p-p\) plot shows that the
CSR model corrects the bias in the other models ·K-S test passes at the
\(5\%\) level for all lines combined Conclusion: The CSR model is
validated

\begin{center}\rule{0.5\linewidth}{0.5pt}\end{center}

\subsection{Process Risk, Parameter Risk and Model
Risk}\label{process-risk-parameter-risk-and-model-risk}

\[\mathrm{Var}(X)=\mathrm{E}_{\theta}\Big[\mathrm{Var}\Big[X\:|\:\theta\Big]\Big]+\mathrm{Var}_{\theta}\Big[\mathrm{E}\Big[X\:|\:\theta\Big]\Big]\quad Total\:Risk=Process\:Risk+Paramet\]

Process Risk - Average variance of outcomes from the expected result

Parameter Risk - Variance due to the uncertainty in the parameters,
reflected in the posterior distributions of the parameters

Model Risk - Risk that we didn't select the ``correct'' model (not
explored directly in Meyers)

\subsubsection{Important observations:}\label{important-observations}

The results in Meyers show that the parameter risk represents the
overwhelming majority of the total risk for the loss datasets.

Model risk can be addressed by weighing together multiple reasonable
models where the weights are additional parameters (Verrall reflects
model risk by weighing together reasonable models

\subsection{Summary of Model Results}\label{summary-of-model-results}

\subsubsection{Incurred Data}\label{incurred-data}

Mack model understates the variability of ultimate losses, particularly
because it assumes independence between accident years, CCL predicts
reasonable distributions of outcomes

\subsection{Paid Data}\label{paid-data}

·Bootstrap ODP, Mack and CCL models predict ultimate losses that are
biased high.

·LIT and CIT allow for a calendar year trend (and correlation in CIT),
but are also biased high

CSR model corrects for the bias and predicts reasonable distributions of
outcomes

\subsection{Summary of Model Results}\label{summary-of-model-results-1}

·p-p Plots \& Histogram Interpret p-p Plots

Kolmogorov -Smirnov (K-S) Test Correlated Chain-Ladder (CCL) Model
Changing Settlement Rate (CSR Model

\begin{center}\rule{0.5\linewidth}{0.5pt}\end{center}

\subsection{Stochastic Loss Reserving Using
GLMs}\label{stochastic-loss-reserving-using-glms}

\section{Key Points}\label{key-points}

The Taylor paper ties in closely with the Mack (1994), Venter Factors
and Shapland papers. I think the most testableinformation comes from the
Chapter 3.Iwould particularly focus on the Mack and CrossClassified
models,Taylor's three theorems relating the models to the chain ladder
method and the GLM representation of the ODP Mack and ODP
Cross-Classified models. Chapter 2 also seems fairly testable for a
question about GLMs in general, model validation or some of the various
formulas.

\subsection{Exponential Dispersion Family
(EDF)}\label{exponential-dispersion-family-edf}

GLMs assume observations Ycome from the exponential dispersion family.
The Tweedie sub-family (and more specifically the Over-Dispersed Poisson
sub-family) are the most important ones for this paper

\subsubsection{Tweedie Sub-Family}\label{tweedie-sub-family}

The Tweedie distribution is defined as an EDF with the following
expected value and variance. The table includes the most important forms
of the Tweedie sub-family that you should be familiar with:

\begin{longtable}[]{@{}lll@{}}
\toprule\noalign{}
Distribution & \(p\) & \(\mu\) \\
\midrule\noalign{}
\endhead
\bottomrule\noalign{}
\endlastfoot
Normal & 0 & \(\theta\) \\
ODP & 1 & \(e^{\theta}\) \\
Gamma & 2 & \(\frac{1}{\theta}\) \\
Inverse Gaussian & 3 & \((-2\theta)^{\frac{1}{2}}\) \\
\end{longtable}

\[\begin{aligned}&V(\mu)=\mu^{p}\quad p\leq0\quad\mathrm{or}\quad p\geq1\\&\boxed{\operatorname{E}[Y]=\mu=\left[\left(1-p\right)\theta\right]^{\frac{1}{1-p}}}\quad\mathrm{for}\:p\neq1\\&\boxed{\operatorname{Var}[Y]=\phi\cdot\mu^{p}}\end{aligned}\]
ODP Distribution
\pandocbounded{\includegraphics[keepaspectratio]{https://storage.simpletex.cn/view/f1M2STs4uwoH0TXtAuICdsH8wYp8rFB1O}}

\subsection{GLMs}\label{glms}

Definition

To specify a GLM, you need to select

·Error distribution (one of the EDF distributions),including index p
Explanatory variables, xs Link function he.g. identity link log-link

The GLM is a regression model and using the specification of the model
and the data (observations and explanatory variables), the GLM will find
the MLE of the \(\beta\) parameters

\begin{center}\rule{0.5\linewidth}{0.5pt}\end{center}

\[\begin{aligned}&\bullet Y_{i}\sim EDF\left(\mu_{i},\phi_{i}\right)\\&\bullet\:b(\mu_{i})=x_{i}^{T}\beta=x_{i1}\beta_{1}+\cdots+x_{in}\beta_{n}\quad or\quad\mu=b^{-1}(X\cdot\beta)\\&\bullet\phi_{i}=\frac{\phi}{w_{i}}\quad w_{i}=weight_{i}\end{aligned}\]

\subsubsection{Covariates: Covariates are the set of predictors or
variables and can be continuous or
categorical.}\label{covariates-covariates-are-the-set-of-predictors-or-variables-and-can-be-continuous-or-categorical.}

·Development year should be used as a categoricalvariate

\section{Goodness of fit and
Deviance}\label{goodness-of-fit-and-deviance}

\[\begin{aligned}\mathrm{of}\:\mu:&\hat{Y}=b^{-1}\left(X\hat{\beta}\right)\\&D^{*}\left(Y,\hat{Y}\right)=2\left(loglikelibood_{Saturated}-loglikelibood_{Model}\right)\end{aligned}\]
Deviance

The saturated modelhas a parameterfor each observation so the
modelcompletely fits the observations

Residuals and Model Validation

\subsubsection{Standardized Pearson
residuals:}\label{standardized-pearson-residuals}

\(R^p = \frac{Y - Y'}{\dot{\sigma}_t}\)

\subsubsection{Model Validation}\label{model-validation}

Check different residual plots against each variable (e.g.~driver age or
development period) to validate the model assumptions.

Standardized Pearson residuals should be random around zero (unbiased)
and have uniform dispersion (homoscedasticity)

\subsubsection{Standardized Deviance
residuals,}\label{standardized-deviance-residuals}

\[R_{i}^{D}=\mathrm{sgn}\Big(Y_{i}-\hat{Y}_{i}\Big)\cdot\sqrt{\frac{d_{i}}{\hat{\phi}}}\quad d_{i}\to\mathrm{Contribution~of~observation~}Y_{i}\mathrm{~to~deviance~}D^{*}\Big(Y,\hat{Y}\Big)\\\mathrm{sgn}(x)\to\mathrm{sign~function~takes~values~-1,~0,~or~1}\]

Standardized Pearson residuals will reproduce any non-normality from the
original data (e.g.~skewed loss data). The standardized Deviance
residuals should be normally distributed, so they're more useful for
mode! assessment.

\subsection{Outliers and Weights.}\label{outliers-and-weights.}

\subsection{Weights: Variance weights can be used to correct
heteroscedasticity. This is similar to how the Shapland
Scale}\label{weights-variance-weights-can-be-used-to-correct-heteroscedasticity.-this-is-similar-to-how-the-shapland-scale}

parameter adjustment works.

\begin{center}\rule{0.5\linewidth}{0.5pt}\end{center}

\subsubsection{Outliers: Observations with very large residuals can skew
the results towards the outlier.An outlier can be
excluded}\label{outliers-observations-with-very-large-residuals-can-skew-the-results-towards-the-outlier.an-outlier-can-be-excluded}

by setting its weight to zero. Careful consideration should be taken to
ensure that it's truly an outlier because removing outliers will limit
the variability in the model.

\subsubsection{Mack Models}\label{mack-models}

Non-Parametric Mack Model

\subsubsection{Assumptions:}\label{assumptions-1}

Accident years are stochastically independent ·For each accident year
\(k\) the cumulative losses \(X_{k,j}\) form a Markov chain ·For each
accident year \(k\) and development periodj
\[\circ\quad\mathrm{E}\Big[X_{k,j+1}\:|\:X_{k,j}\Big]=f_{j}X_{k,j}\\\circ\quad\mathrm{Var}\Big[X_{k,j+1}\mid X_{k,j}\Big]=\sigma_{j}^{2}X_{k,j}\]

Note that these are essentially the same assumptions as the normal Mack
model assumptions. This model is:

Stochastic - Because it considers both the expected value and variance
of observations

Non-Parametric - It doesn't consider the distribution (e.g.~Gamma) of
the observations

\subsubsection{Results of theMackModel}\label{results-of-themackmodel}

1.The conventional chain ladder LDF estimators, \(\hat{f}_{j}\) ,are
a.Unbiased b.Minimum variance among estimators that are unbiased linear
combinations of the triangle's age-to-age factors, \(\hat{f}_{k,j}\) 2.
The conventional chain ladder reserve estimator, \(\hat{R}_{k}\) , is
unbiased

\subsubsection{Parametric Mack Model}\label{parametric-mack-model}

\subsubsection{Changed assumptions from
above:}\label{changed-assumptions-from-above}

\[Y_{_{k,j+1}}\mid X_{_{k,j}}\sim EDF\left(\theta_{_{k,j}},\phi_{_{k,j}};a,b,c\right)\]
·Variance assumption is removed -Variance is driven by the selected EDF
distribution

\begin{center}\rule{0.5\linewidth}{0.5pt}\end{center}

\subsubsection{Theorem 1}\label{theorem-1}

For losses in an nxn triangle with observations subject to the EDF Mack
model assumptions:

a.If the original Mack variance assumption also holds, then the MLEs of
the \(f_{j}\) parameters are the conventional chain ladder LDF
estimators, \(\hat{f}_{j}\) , and these are unbiased estimators.

b.If the model is restricted to the ODP Mack model and if the dispersion
parameters are just column dependent, \(\phi_{k,j}=\phi_{j}\) , then the
weighted average chain ladder LDFs, \(\hat{f}_{j}\) , are minimum
variance unbiased estimators (MVUEs) c.For the same conditions as (b),
the cumulative loss estimates, \(\hat{X}_{k,j}\) , and reserve
estimates, \(\hat{R}_k\) , are also MVUEs.

\subsubsection{Implication of Theorem 1: This theorem means that the
conventional chain ladder estimates and forecasts are optimal estimators
(both}\label{implication-of-theorem-1-this-theorem-means-that-the-conventional-chain-ladder-estimates-and-forecasts-are-optimal-estimators-both}

MLE and MVUEs). This is a stronger implication than the original Mack
model since it shows that the conventional chain ladder LDF estimators,
\(\hat{f}_{j}\) , are minimum variance for all unbiased estimators, not
just of the linear combinations of the age-to-age factors.

\subsubsection{Cross-Classified Models}\label{cross-classified-models}

Cross-Classified Model

\subsubsection{Assumptions:}\label{assumptions-2}

·The random variables \(Y_{k,j}\) are stochastically independent

·For each accident year \(k\) and development period
。\(Y_{k,j}\sim EDF\big(\theta_{k,j},\phi_{k,j};a,b,c\big)\)
。\(\operatorname{E}\biggl[Y_{k,j}\biggr]=\alpha_{k}\beta_{j}\) o∑β,=1

\subsection{Theorem 2}\label{theorem-2}

For losses in an nxn triangle with observations subject to the ODP
Cross-Classified model assumptions and the following assumptions:

· \(Y_{k,j}\) is restricted to an ODP distribution

·The dispersion parameters are identical for all cells:
\(\phi_{k,j}=\phi\)

Then the MLE fitted values \(\hat{Y}_{k,j}\) are the same as those from
the conventional chain ladder method.

\subsubsection{Theorem 3}\label{theorem-3}

If the Theorem 2 assumptions hold and the fitted values
\(\hat{Y}_{k,j}\) and reserve estimates \(\hat{R}_k\) are corrected for
bias, then they are MVUEs of \(Y_{k,j}\) and \(R_{k}\)

\begin{center}\rule{0.5\linewidth}{0.5pt}\end{center}

These two theorems are similar to Theorem1 about the ODP Mack model and
mean that:

1.Forecasts from the ODP Mack and ODP Cross-Classified models are
identical and the same as those from the chain ladder method despite the
different formulations.

2.We can get forecasts for the ODP Cross-Classified model without
considering the model directly and working as if the model was an ODP
Mack model.

\subsubsection{GLM Representation of Chain Ladder
Models}\label{glm-representation-of-chain-ladder-models}

ODP Mack Model GLM

To represent the ODP Mack Model as a GLM, we model the triangle's
age-to-age factors, \(\hat{f}_{k,j}\)

\[age-to-age_{k,j}=\frac{CumLoss_{AY,d+1}}{CumLoss_{AY,d}}\]

\pandocbounded{\includegraphics[keepaspectratio]{https://storage.simpletex.cn/view/fC4epcxy02dADDdSpcULn7WExt4qAriHE}}

\[\mathbf{X}=\left(\begin{array}{ccc}1&0&0\\\\0&1&0\\\\0&0&1\\\\1&0&0\\\\0&1&0\\\\1&0&0\end{array}\right)\quad\beta=\left[\begin{array}{c}f_1-1\\\\f_2-1\\\\f_3-1\end{array}\right]\]

\[
\begin{aligned}
&\boxed{w_{k,j}=X_{k,j}} \\
&weight_{AY,dev}=CumLoss_{AY,dev}
\end{aligned}
\]

中\(\hat{f}_{2,1}-1\) \(\hat{f}_{1,3}-1\) \(\hat{f}_{1,2}-1\)
\[\mu=h^{-1}(\mathrm{X}\beta)\] \(b=\) identity function

\[\hat{f}_{k,j}-1\mid X_{k,j}\sim ODP\Bigg(f_{j}-1,\frac{\phi_{j}}{X_{k,j}}\Bigg)\]

The output of the GLM will be the best-fit parameter estimates of the
development factors, \(f_{j}-1\)

\begin{center}\rule{0.5\linewidth}{0.5pt}\end{center}

\subsubsection{ODP Cross-Classified Model
GLM}\label{odp-cross-classified-model-glm}

\pandocbounded{\includegraphics[keepaspectratio]{https://storage.simpletex.cn/view/ff5D5HKboxhzmWNyPiHYP81e0ODgfi0OH}}

\[
\begin{aligned}
&\mu_{k,j}=\exp\Bigl(\ln\alpha_{k}+\ln\beta_{j}\Bigr) \\
&b=\log\operatorname*{lim}\mathrm{function} \\
&Y_{k,j}\sim ODP\big(\mu_{k,j},\phi\big)
\end{aligned}
\]

Reliance on Recent Experience Years

Like normal reserving, we may only want to use the latest m experience
years in the model. To do this, we simply set the weight of the
observations before the latest \(m\) experience years to zero in the
model.

\subsection{Recipes for Calculation
Problems}\label{recipes-for-calculation-problems-10}

ODP Cross-Classified Model Parameters GLM Representation of Chain Ladder
Modelg Re-Normalizing ODP Cross-Classified Parameter

\begin{center}\rule{0.5\linewidth}{0.5pt}\end{center}

\subsection{Predictive Distributions for Reserves which Incorporate
Expert
Opinion}\label{predictive-distributions-for-reserves-which-incorporate-expert-opinion}

\subsubsection{Overview}\label{overview-11}

In traditional loss reserving with deterministic methods, it's common to
select LDFs or select an expected loss ratio for theBF method.These are
ways to incorporate expert opinion.However,if we want to calculate a
reserve range using a traditional bootstrap model or Mack model, making
LDF selections may compromise some of the model assumptions.

A key benefit of using a Bayesian approach (compared to a Mack or
bootstrap model) is that we can naturally incorporate expert opinion
within a stochastic model and get a reserve range without compromising
the underlying assumptions.

Examples of when to incorporate expert knowledge or opinion.

·Change in payment pattern due to a change in company policy

Change in benefits due to newlaws,requiring adjustment to LDFs

\subsubsection{Areas where expert knowledge might be
used}\label{areas-where-expert-knowledge-might-be-used}

Selecting LDFs for the chain ladder method OOverriding LDFs in a
particular row (accident year) OUsing only n-years to estimate the LDFs

·Selecting expected loss for the BF method

\subsection{Different Stochastic Models for Chain
Ladder.}\label{different-stochastic-models-for-chain-ladder.}

Each of the models below give the same reserve result as the chain
ladder method,but they differ in form and may have different variances.

\subsubsection{Mack Model}\label{mack-model}

\(\mathbb{E}\left[D_{i,j}\right]=\lambda_j\cdot D_{i,j-1}\)

\(\text{Var}\left(D_{i,j}\right)=\sigma^2_j\cdot D_{i,j-1}\)

\[
\begin{gathered}
\operatorname{E}\bigg[\:Loss_{AY,k}\bigg]\:=\:LDF_{k}\cdot Loss_{AY,k-1} \\
\mathrm{Var}\left(Loss_{AY,k}\right)=\sigma_{j}^{2}\cdot Loss_{AY,k-1}
\end{gathered}
\]

\subsection{Advantages:}\label{advantages-8}

·Easy to implement - Parameter estimates and prediction errors can be
calculated in a spreadsheet.

\subsubsection{Disadvantages}\label{disadvantages-6}

There is no predictive distribution, because a distribution isn't
specified.

\begin{center}\rule{0.5\linewidth}{0.5pt}\end{center}

·Need to estimate separate parameters, \(\sigma_{j}^{2}\) , for the
variance apart from the LDFs.

\subsubsection{Over-Dispersed Poisson
Models}\label{over-dispersed-poisson-models}

\subsection{GLM approach(similar to Shapland GLM
Bootstrap)}\label{glm-approachsimilar-to-shapland-glm-bootstrap}

\(\mathbb{E}\left[C_{i,j,t}\right]=m_{ij}=e^{\epsilon\alpha_i\alpha_j}\)

\[\mathrm{E}\Big[IncLoss_{_{AY,k}}\Big]=e^{c+\alpha_{i}+\beta_{j}}\]

\subsubsection{Row-Column Form:}\label{row-column-form}

and \(\sum y_{j}=1\)

\(E[C_{ij}] = x,y\)

E{[}IncLossg= Row FactoryCol Factor

\(X_i-\) Expected ultimate loss for accident year \(i\)up to thelast
development period of the triangle \(y_{j}-96\) \% 96 of ultimate loss
emerging in development period

\subsection{Advantages:}\label{advantages-9}

·Doesn't necessarily break down if there are some negative incremental
values.

·It gives the same reserve estimate as the chain ladder method

·It's more stable than the log-normal model of Kremer.

\subsection{Disadvantages:}\label{disadvantages-7}

Connection to the chain ladder method is not immediately apparent.

\subsubsection{Over-Dispersed Negative
Binomia}\label{over-dispersed-negative-binomia}

This model differs in form from the ODP model, but produces the same
predictive distribution

\(\mathbb{E}\left[C_{i,j}\right]=\left(\lambda_j-1\right)D_{i,j-1}\)

\(\operatorname{Var}\left(C_{i,j}\right)=\varphi\cdot\lambda_j\cdot\left(\lambda_j-1\right)\cdot D_{i,j-1}\)

\subsection{Advantages:}\label{advantages-10}

Doesn't necessarilybreak down if there are some negative incremental
values (but the sum of the column needs to be positive)

·It gives the same reserve estimate and has the same form as the chain
laddermethod

\subsection{Disadvantages:}\label{disadvantages-8}

Column sums of incremental losses must be positive (or variance would be
negative)

\subsubsection{Normal Distribution}\label{normal-distribution}

\(E[C_{i,j}] = (\lambda_j - 1)D_{j,i-1}\)
\[\text{Var}(C_{i,j}) = \varphi_j D_{j,i-1}\]
\[\mathrm{E}\Big[IncLoss_{_{AY,k}}\Big]=\Big(LDF_{_k}-1\Big)\cdot Loss_{_{AY,k-1}}\]

\[\mathrm{Var}\Big(IncLoss_{AY,k}\Big)=dispersion_{k}\cdot Loss_{AY,k-1}\]

\begin{center}\rule{0.5\linewidth}{0.5pt}\end{center}

\subsection{Advantages:}\label{advantages-11}

:Can handle negative incremental losses

\subsubsection{Disadvantages:}\label{disadvantages-9}

·Need to estimate separate parameters, \(\varphi_{j}\) , for the
variance apart from the LDFs.

\subsection{Prediction Error}\label{prediction-error}

Given loss data,we can use one of the models above to create a
distribution of future losses.The expected value of the distribution is
our prediction, the central estimate. The root mean square error (RMSEP)
is the prediction error, incorporating both process variance and
parameter (estimation) variance.

\[\boxed{prediction\:variance=process\:variance+estimation\:variance}\]

\subsubsection{Prediction Error vs.Standard
Error}\label{prediction-error-vs.standard-error}

Prediction error accounts for uncertainty in the parameter estimation
(estimation variance) and the variability of future losses (process
variance). Standard error only accounts for estimation variance.

\[predictionerror=\sqrt{process\:variance+estimation\:variance}standard\:error=\sqrt{estimation\:variance}\]

\subsection{Incorporating Expert Opinion in Chain
Ladder.}\label{incorporating-expert-opinion-in-chain-ladder.}

The over-dispersed negative binomial model can allow for a separate LDF
parameter for each accident year and development period. This
flexibility allows us to incorporate expert opinion in selecting LDFs.

\subsubsection{Incorporating Expert Opinion to Override Specific
LDFs}\label{incorporating-expert-opinion-to-override-specific-ldfs}

·For each LDF parameter that is selected, specify the prior distribution
of \(\lambda_{i,j}\)
\[\mathrm{E}\Big[\lambda_{i,j}\Big]=LDF^{selection}\quad\mathrm{Var}\Big(\lambda_{i,j}\Big)-\mathrm{set~to~be~relatively~small}\]

·Group the other LDF parameters by development period to reflect the
chain ladder method,by setting \(\lambda_{i,j}=\lambda_{j}\) and using a
large variance for the prior distribution.

\subsubsection{Incorporating Expert Opinion to Use n-year
LDFs}\label{incorporating-expert-opinion-to-use-n-year-ldfs}

for the mostrecent n calendar year diagonals
\[\lambda_{i,j}=\lambda_{j}\\\lambda_{i,j}=\lambda_{j}^{*}\] for all
diagonals prior to the latest ndiagonals

Set the prior distributions of the LDF estimates \(\lambda_{j}\) and
\(\lambda_{j}^{*}\) with large variances so that they're estimated from
the data.

\subsubsection{How Variance of the Prior Distribution Impacts the Model
Result}\label{how-variance-of-the-prior-distribution-impacts-the-model-result}

·``Vague'' priors (large variance)-Result is close to what the data
estimates (chain ladder method) ``Strong''priors (small variance)-Result
is drawn toward the mean of the prior distribution

\begin{center}\rule{0.5\linewidth}{0.5pt}\end{center}

\subsection{Bayesian Model for
Bornhuetter-Ferguson}\label{bayesian-model-for-bornhuetter-ferguson}

The Bayesian model of the BF method isbased on the over-dispersed
Poisson distribution (row-col form) For the BFmodel, select the expected
ultimate loss by accident year by setting a prior distribution for each
of the row parameters, \(X_{i}.\)

\pandocbounded{\includegraphics[keepaspectratio]{https://storage.simpletex.cn/view/fTGQhStTqCsMVGQlslxgXFS5L9GWixKeE}}

Credibility-Weighted Bayesian Model for the BF Method

TheBayesian model is a credibility model between the chain ladder and
Bornhuetter-Ferguson methods. We can adjust the credibility by changing
\(\beta_{i}\)

:Increasing \(\beta_{i}\) causes the variance of the prior distribution
to decrease (indicating higher confidence in the selected ultimate loss)
This causes \(Z_{i,j}\) to decrease placing more weight on the BF method

\[
\begin{gathered}
\operatorname{E}\left[C_{i,j}^{cL}\right]=\left(\lambda_{j}-1\right)\cdot D_{i,j-1} \operatorname{E}\bigg[C_{i,j}^{BF}\bigg]\:=\:M_{i}\cdot96IncPaid_{j} \\
Z_{i,j}=\frac{p_{j-1}}{\beta_{i}\cdot\varphi+p_{j-1}} \\
\operatorname{E}\bigg[C_{i,j}^{\sigma\alpha d}\bigg]=Z_{i,j}\cdot\operatorname{E}\bigg[C_{i,j}^{CL}\bigg]+(1-Z_{i,j})\cdot\operatorname{E}\bigg[C_{i,j}^{BF}\bigg]
\end{gathered}
\]

Fully Stochastic Bayesian Model for the BF Method

The model above is stochastic in the row parameters, \(X_{j}\), but uses
the fixed LDFs directly

The fully stochastic model uses an improper gamma prior distribution for
the column parameters and the prior distribution shown above for the row
parameters, \(X_{i}.\) The goal here is to have a model that is fully
stochastic in both row and column parameters.

\subsubsection{How Variance of the Prior Distribution Impacts the Model
Results}\label{how-variance-of-the-prior-distribution-impacts-the-model-results}

Vaguepriors for \(X_{j}\) (large variance)-Result is close to the chain
ladder method ``Strong'' priors for \(X_{i}\) (small variance) - Result
is close to the Bornhuetter-Ferguson method

The flexibility of the fully stochastic modelmeans we can replicate the
chain ladder method,the BFmethod or a weighting between the two methods
by adjusting the variance of the prior distributions of the row
parameters.

\subsubsection{Advantage of a Stochastic
Approach}\label{advantage-of-a-stochastic-approach}

We can produce thefull predictive distribution of the reserve,not just
the point estimate or the mean and standard deviation

\begin{center}\rule{0.5\linewidth}{0.5pt}\end{center}

\subsection{Recipes for Calculation
Problems}\label{recipes-for-calculation-problems-11}

Incorporating Expert Opinion: Chain Ladder Bayesian Moade fortheBE
Methode Estimated Reserve from Fully Stochastic BF

\begin{center}\rule{0.5\linewidth}{0.5pt}\end{center}

\subsection{A Framework for Assessing Risk
Margins}\label{a-framework-for-assessing-risk-margins}

\subsubsection{Overview}\label{overview-12}

The main goal of the Marshall paper is to answer the question of how to
set an appropriate risk margin for insurance liabilities.For this paper,
make sure to focus on the sources of uncertainty (independent risk and
internal/external systemic risk). Also, make sure you thoroughly
understand the risk margin calculation and the different types of
additional analyses.

\section{Preparing the Claims
Portfolio}\label{preparing-the-claims-portfolio}

Splitting the Claims Portfolio to Valuation Classes

It's preferable to split the claims portfoliointo the same valuation
classes that are used in the reserve analysis to determine central
estimates.

Splitting Valuations Classes to Claims Groups

There should be homogeneity within a claims group and different behavior
between claims groups (e.g.~split the Home portfoliobetween CAT and
non-CAT claims)

\subsection{Sources of Uncertainty}\label{sources-of-uncertainty}

Independent Risk - Risks arising from randomness inherent in the
insurance process

Systemic Risk -Risks that are potentially common across valuation
classes or claims groups

:Internal Systemic Risk - Uncertainty because the valuation models are
imperfect representations of the insurance process. These are risks
internal to the liability valuation process. ·External Systemic Risk
-Risks external to the liability valuation process

\subsubsection{Sources of Independent
Risk}\label{sources-of-independent-risk}

·Random component of parameter risk -Uncertainty in selecting
appropriate parameters due to the randomness of insurance. Random
component of process risk-Pure effect of randomness due to the insurance
process

Sources of Internal Systemic Risk

Specification Error -Error from the inability to make a model that fully
represents the insurance process

Parameter Selection Error - Error that we can't adequately measure all
the predictors (parameters) of claim cost outcomes and their trends

Data Error -Error due to poor or unavailable data that's needed for the
reserve analysis

\subsubsection{Sources of External Systemic
Risk}\label{sources-of-external-systemic-risk}

Economic and Social Risks -Normal inflation and social/environmental
trends

\begin{center}\rule{0.5\linewidth}{0.5pt}\end{center}

Legislative, Political and Claims Inflation Risk - Changes to the
legislative or political environment affecting claims valuation
portfolios and changes to claims settlement levels Claim Management
Process Change Risk-Changes in the claims department that impact
reporting/payment patterns or the estimating of case reserves Expense
Risk - Uncertainty in expenses for managing the liabilities Event Risk
(Catastrophes) - Uncertainty from losses due to natural or man-made
catastrophes ·Latent Claim Risk - Uncertainty from losses that may arise
in the future from a source that's not considered covered
(e.g.~asbestos) Recovery Risk - Uncertainty in reinsurance (or
non-reinsurance) recoveries

\subsubsection{Quantitative vs.Qualitative
Analysis}\label{quantitative-vs.qualitative-analysis}

Quantitative analysis can only reflect uncertainty in historical
experience and can't adequately reflect all possible sources of future
uncertainty ·We also need qualitative analysis (judgment) to make sure
we reflect all sources of uncertainty

Quantitative modeling work best to analyze sources of independent risk
and past episodes of external systemic risk. Quantitative modeling needs
to be supplemented with other quantitative and qualitative analyses to
reflect internal systemic risk and future external systemic risk that
differs from the past

\subsubsection{Correlation Effects}\label{correlation-effects}

Correlation for Independent Risks

Independent risks are uncorrelated with other sources of uncertainty
both within and between valuation classes.

\subsubsection{Correlation for Internal Systemic
Risks}\label{correlation-for-internal-systemic-risks}

Uncorrelated with independent risk and external systemic risk sources
·Correlated between classes or between outstanding claim and premium
liabilities due to: OSame actuary effect - Effect of common valuation
approaches for different valuation classes OLink between premium
liability methodology and outcomes from outstanding claim. valuation

Correlation for External Systemic Risks

Uncorrelated with independent, internal systemic risk and generally
uncorrelated with other. external systemic risk categories (e.g.~event
risk is uncorrelated with economic/social risk) Correlated between
classes or between outstanding claim and premium liabilities in the same
risk category ``It is possible that external systemic risk categories
may be partially correlated either within or between valuation classes

\begin{center}\rule{0.5\linewidth}{0.5pt}\end{center}

\subsubsection{Quantitative Methods for Assessing
Correlation}\label{quantitative-methods-for-assessing-correlation}

Quantitative techniques are complex and require lots of data - time and
effort outweigh benefits Correlations would be driven by past
correlations, but correlations for external systemic risk sources may be
different than past episodes. ·It's difficult to separate past
correlations between independent and systemic risks. Internal systemic
risk can't be modeled with standard correlation techniques ·Results are
unlikely to be aligned with the framework (split between independent and
internal/external systemic risk

\subsection{Risk Margin Calculation.}\label{risk-margin-calculation.}

To calculate the risk margin,we calculate the CoVs separately for
independent risk,internal systemic risk and external systemic risk.
Then, we consolidate the CoVs assuming independence between the sources
of risk and calculate the risk margin (assuming the normal or lognormal
distribution)

\subsection{Additional Analysis}\label{additional-analysis}

Sensitivity Testing

Goal:

Gain insights about the sensitivity of the final risk margin to key
assumptions.

Do this by varying each key assumption (the CoVs and correlations) and
monitor the impact on the risk margin.Review key assumptions that have
significant impact on the risk margin.

\subsection{Scenario Testing}\label{scenario-testing}

Goal:

Tie the risk margin to a set of valuation outcomes. We adjust the
assumptions used for the central estimate in order to align the central
estimate with the assessed provisions including the risk margin

Scenario testing asks the question: What scenario (e.g.higher frequency
or severity) would result in the central estimate increasing to the
current central estimate plus the risk margin?

\subsubsection{Internal Benchmarking}\label{internal-benchmarking}

Goal: Compare the proposed CoVs to one another to check for internal
consistency and reasonableness.

\subsubsection{Independent Risk CoV
Selection:}\label{independent-risk-cov-selection}

Portfolio size: Larger portfolios have lower Co Vs due to lower
volatility from random effects ·Length of Claim Runoff: Longer tailed
lines have higher CoVs due to more time for random effects to have an
impact

Implications:

Long-tailed portfolios: Premium Liability CoV should be higher than
Outstanding Claim CoV

\begin{center}\rule{0.5\linewidth}{0.5pt}\end{center}

\subsection{Internal Systemic Risk CoV
Selection:}\label{internal-systemic-risk-cov-selection}

·If template models are used for similar valuation classes,we would
expect similar CoVs ·If similar valuation methodology is used on both
short and long-tail classes, then we would expect a higher CoV for the
long-tail class.

\subsubsection{External Systemic Risk CoV
Selection}\label{external-systemic-risk-cov-selection}

CoVs should be higher for long-tail portfolios,except for event risk for
Property and liability risk forHome

\subsubsection{External Benchmarkinc}\label{external-benchmarkinc}

Goal:

Compare thereasonableness of CoVs and risk margins to those from
external sources

Shouldn't be relied on instead of analysis,but may be useful when there
is little information available for analysis, especially for independent
risk.

\subsection{Hindsight Analysis.}\label{hindsight-analysis.}

Goal: Compare the past reserve estimates of liabilities against the
latest view of the same liabilities to review the

actual volatility in thepast

\subsection{Mechanical Hindsight
Analysis}\label{mechanical-hindsight-analysis}

3.Apply the chain ladder method to data as of the valuation date to
calculate a current unpaic estimate. 4.Iteratively, remove a diagonal of
data and apply the same method to calculate unpaid estimates for prior
valuation dates. 5.Compare the past estimates to the current estimate
for the same accident years.The relevant payments made between the past
valuation dates and the current valuation date should be added to the
current unpaid estimate.

\subsection{Independent Risk
Assessment}\label{independent-risk-assessment}

The goal of the independent risk assessment is to use a model to``fit
away'' past systemic episodes in order to analyze the residual
volatility in order to get the CoVs for independent risk.

\subsubsection{Methods to Analyze Independent
Risk}\label{methods-to-analyze-independent-risk}

·Mack method (Mack 1994 paper) Bootstrapping (Shapland paper Stochastic
Chain Ladder ·GLM techniques

\begin{center}\rule{0.5\linewidth}{0.5pt}\end{center}

:Bayesian techniques (Meyers paper)

\subsection{Balanced Scorecard (Internal Systemic Risk
Assessment}\label{balanced-scorecard-internal-systemic-risk-assessment}

Goal:Assess the adequacy of the modeling infrastructure and its ability
to reflect and predict the underlying insurance process.

\subsection{Potential Risk Indicators}\label{potential-risk-indicators}

Page 33 in the paper has a full list of the potential risk
indicators.Below are a few examples

\subsection{Specification Error:}\label{specification-error}

·Number of independent models used

·Range of results produced by the models

·Checks made on reasonableness of results

\subsubsection{Parameter Selection
Error:}\label{parameter-selection-error}

Best predictors have been identified

·Best predictors are stable over time (or change due to process changes)
.Value of the predictors used (predictors used are close to the best
predictors)

\subsubsection{Data Error}\label{data-error}

Good knowledge of past processes affecting predictors Extent,
timeliness, consistency and reliability of information from business
·There are appropriate reconciliations and quality control for the data

\subsection{Reviewing the CoV Scale.}\label{reviewing-the-cov-scale.}

Minimum CoV for best practice is unlikely to be less than \(5\%\)
Maximum CoV for worst practice could be greater than \(20\%\)
(e.g.~single, aggregated model with limited data) Scale should norbe
linear -Model improvements show diminishing returns OImprovement from a
poor to fair model is greater than from a fair to good model CoVs for
long-tail lines are higher than short-tail CoVs for the same score 0It's
more difficult to model the underlying process of a long-tail line and
predictors are less stable ·It's reasonable to use the same scale for
outstanding claim liabilities and premium liabilities

\subsection{Types of External Systemic
Risk}\label{types-of-external-systemic-risk}

We can identify key potential sources of systemic risk through
discussions with business experts as part of the valuation process.
These discussions should consider:

\begin{center}\rule{0.5\linewidth}{0.5pt}\end{center}

Underwriting and risk selection Pricing Claims Management Expense
Management Emerging portfolio trends ·The environment in which the
portfolio operates

CoVs for each external systemic risk category should be selected using a
mix of quantitative analysis and qualitative judgment

Economicand Social Risk

Includes normal inflation (CPI) and economic conditions (unemployment,
GDP growth, \ldots)

Legislative, Political and Claims Inflation Risk

These risks are likely to be more significant for long-tail lines. Some
examples include

·Impact of recent legislation or changes in court interpretation
Potential future legislation that impacts losses retrospectively
·Precedent setting in courts ·Changes in medical technology costs or
legal costs ·Systemic shifts in large claim frequency or severity

\subsubsection{Claim Management Process Change
Risk}\label{claim-management-process-change-risk}

Changes can impact all valuation classes, but are more relevant for
outstanding claims liabilities than premium liabilities. It's important
to discuss current and future process changes with claims management and
how changes could impact reporting/payment patterns, closing and
reopening rates, and case reserves.

\subsubsection{Expense Risk}\label{expense-risk}

Expense risk is generally a small contributor to total external systemic
risk

\subsubsection{Event Risk Catastrophes}\label{event-risk-catastrophes}

Catastrophes have the most significant impact on property lines of
business.

Sources to help quantify the impact of event risk for premium
liabilities:

·Past experience from event risk losses ·CAT models Reinsurance
intermediaries

\begin{center}\rule{0.5\linewidth}{0.5pt}\end{center}

\subsection{Latent Claim Risk}\label{latent-claim-risk}

Latent claim risk may be material for workers compensation and liability
classes,but is negligible for most valuation classes. This risk has low
probability, but high potential severity (e.g.~asbestos). Discussions
with underwriters can help to identify sources of risk

\subsubsection{Recovery Risk}\label{recovery-risk}

Likely to be insignificant for most valuation classes, except
potentially for third party recoveries for Auto. Reinsurance recovery
risk could be driven by catastrophes.

\subsection{Recipes for Calculation
Problems}\label{recipes-for-calculation-problems-12}

Risk Margin Calculation Internal Risk Balanced Scorecard

\end{document}
