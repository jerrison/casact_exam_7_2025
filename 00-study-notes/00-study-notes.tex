\documentclass{article}
\usepackage{graphicx} % Required for inserting images
\usepackage{amsmath}
\usepackage[margin=1in]{geometry}
\usepackage{colortbl}
% \usepackage{xcolor}
\usepackage[table]{xcolor}
\usepackage{cancel}
\usepackage[parfill]{parskip}

\title{CASACT Exam 7 Notes}
\author{Jerrison Li }
\date{January 2025}

\begin{document}

\maketitle

\section{Benktander Method}

\subsection{General Relationship Between Reserve \& Ultimate Loss Estimates }

Suppose that $C_k$ is the actual claims amount paid after \textit{k} years of
development. Given a \textbf{reserve estimate} $\widehat{R}$ and
\textbf{ultimate loss estimate} $\widehat{U}$, we have the following general
relationship:

\begin{equation}
  \widehat{U} = C_k + \widehat{R}
\end{equation}

This relationship always holds. Note that $C_k$ is the cumulative paid amount.

\subsection{Bornhuetter/Ferguson (BF) Method }

The \textbf{Bornhuetter/Ferguson (BF) Method} estimates reserves based on an
\textbf{a priori} expectation of ultimate losses. Mathematically:

\begin{equation}
  R_{BF} = q_k * U_0
\end{equation}

where:

\begin{itemize}
  \item $R_{BF}$ is the \textbf{reserve estimate}
  \item $q_k = 1 - \frac{1}{CDF}$ is the proportion of the ultimate claims
    amount which is expected to remain unpaid after \textit{k} years
    of development
  \item $U_0$ is the a priori expectation of ultimate losses
\end{itemize}

\bigskip

Since $R_{BF}$ uses $U_0$, it assumes that the current claims amount $C_k$ is
\textbf{NOT predictive of future claims}.

Using the general relationship described above, we obtain the \textbf{BF
ultimate loss}:

\begin{equation}
  U_{BF} = C_k + R_{BF}
\end{equation}

\subsection{Chain Ladder Method }

The \textbf{Chain Ladder Method} estimates ultimate losses and reserves based
on claims to date. In other words, it assumes that the current claims amount
$C_k$ is \textbf{fully predictive of future claims}. Mathematically:

\begin{equation}
  U_{CL} = \frac{C_k}{p_k}
\end{equation}

\begin{equation}
  R_{CL} = q_k * U_{CL}
\end{equation}

\subsection{Benktander Method }

Since the CL and BF methods represent extreme positions, where the CL method
fully believes $C_k$ and the BF method does not rely on $C_k$ at all, Gunnar
Benktander replaced $U_0$ with a credibility mixture:

\begin{equation}
  U_c = c * U_{CL} + (1 - c) * U_0
\end{equation}

where $c$ is the credibility weight.

\bigskip

As the claims $C_k$ develop, credibility should increase, Benktander proposed
the following:

\begin{itemize}
  \item Set $c = p_k$
  \item Set $R_{GB} = R_{BF} * \frac{U_{pk}}{U_0}$
\end{itemize}

\begin{equation}
  \begin{aligned}
    R_{GB} &= \color{blue}{R_{BF}} &* \color{black}\frac{U_{pk}}{U_0} \\
    &= (\color{blue}{q_k * \cancel{U_o}}) &*
    \color{black}\frac{U_{pk}}{\cancel{U_o}} \\
    &= q_k * U_{pk}
  \end{aligned}
\end{equation}

\subsubsection{BF Method as a Credibility-Weighted Average}

Using our credibility mixture with $c = p_k$, we can show the following:

\begin{equation}
  \begin{aligned}
    U_{pk} &= p_k * U_{CL} + (1 - p_k) * U_0 \\
    &= p_k * U_{CL} + q_k * U_0 \\
    &= C_k + R_{BF} \\
    &= U_{BF}
  \end{aligned}
\end{equation}

then

\begin{equation}
  \begin{aligned}
    R_{GB} &= q_k * U_{BF} \\
  \end{aligned}
\end{equation}

Hence, the \textbf{BF method} is a credibility-weighted average of the CL method
and the a priori expectation.

\subsubsection{Benktander Method as a Credibility-Weighted Average}

The \textbf{Benktander method} is a credibility-weighted average of the CL and
BF methods:

\begin{equation}
  \begin{aligned}
    U_{GB} &= C_k & + R_{GB} \\
    &= \color{blue}{C_k} & + q_k * U_{BF} \\
    &= \color{blue}{(1-q_k) * U_{CL}} & + q_k * U_{BF} \\
    % &= \color{blue}{p_k * U_{CL}} & + q_k * q_k * U_0 \\
    % &= \color{blue}{(1-q_k) * U_{CL}} & + q_k * q_k * U_0 \\
    % &= \color{blue}{(1-q_k) * U_{CL}} & + q_k^2 * U_0 \\
  \end{aligned}
\end{equation}

The \textbf{Benktander reserve} is also a credibility-weighted average of the CL
and BF methods. This was proposed by Esa Hovinen (1981):

\begin{equation}
  \begin{aligned}
    R_{EH} &= \color{red}(1-q_k) \color{blue}* R_{CL} \color{black}+
    q_k * R_{BF} \\
    &= \color{red}p_k * \color{blue}q_k * U_{CL} \color{black}+ q_k *
    q_k * U_0 \\
    &= \color{red}p_k * \color{blue}q_k * U_{CL} \color{black}+
    (1+p_k) * q_k * U_0 \\
    &= q_k * [p_k * U_{CL} + (1-p_k) * U_0] \\
    &= q_k * U_{pk} \\
    &= R_{GB}
  \end{aligned}
\end{equation}

We can also express the Benktander method as a credibility-weighted average of
the CL method and the a priori expectation:

\begin{equation}
  \begin{aligned}
    U_{GB} &= C_k  + R_{GB} \\
    &= \color{blue}{U_{CL} - \color{red}R_{CL}}  + q_k * U_{pk} \\
    &= \color{blue}{U_{CL} - \color{red}q_k * U_{CL}}  + q_k *
    ([1-q_k] * U_{CL} + q_k * U_0) \\
    &= \color{blue}{U_{CL} - \cancel{\color{red}q_k * U_{CL}}}  +
    \cancel{q_k * U_{CL}} - q_k^2 * U_{CL} + q_k^2 * U_0 \\
    &= \color{blue}{U_{CL}}   \color{black}- q_k^2 * U_{CL} + q_k^2 * U_0 \\
    &= (1-q_k^2) * U_{CL} + q_k^2 * U_0 \\
    &= U_{1-q_k^2}
  \end{aligned}
\end{equation}

\subsubsection{Iterative Relationships}

As shown above, the Benktander reserve is obtained by applying the BF method
twice, first to $U_0$ (which produces $R_{BF}$) and then to $U_{BF}$ (which
produces $R_{GB}$). Hence, the Benktander method is also called the \textbf{
iterated BF method}.

Using a starting point of $U^{(0)} = U_0$ and iteration rules $R^{(m)} = q_k *
U^{(m)}$ and $U^{(m+1)} = C_k + R^{(m)}$, we obtain the following iterative
relationships:

\begin{equation}
  \begin{aligned}
    U^{(m)} &= (1 - q_k) * U_{CL} + q_k^m * U_0 \\
    R^{(m)} &= (1 - q_k) * R_{CL} + q_k^m * R_{BF} \\
  \end{aligned}
\end{equation}

These relationships lead to the following:

\begin{itemize}
  \item $U^{(0)} = (1 - q_k^0) * U_{CL} + q_k^0 * U_0 = U_{0}$
  \item $R^{(0)} = (1 - q_k^0) * R_{CL} + q_k^0 * R_{BF} = R_{BF}$
  \item $U^{(1)} = (1 - q_k) * U_{CL} + q_k * U_0 = C_k + R_{BF} = U_{BF}$
  \item $R^{(1)} = (1 - q_k) * R_{CL} + q_k * R_{BF} = R_{GB}$
  \item $U^{(2)} = (1 - q_k^2) * U_{CL} + q_k^2 * U_0 = U_{GB}$
  \item \ldots
  \item $U^{(\infty)} = (1 - q_k^\infty) * U_{CL} + q_k^\infty * U_0 = U_{CL}$
  \item $R^{(\infty)} = (1 - q_k^\infty) * R_{CL} + q_k^\infty *
    R_{BF} = R_{GB}$
\end{itemize}

\subsection{Benktander vs. BF vs. CL }
The Benktander method is superior to the BF and CL methods for the following
reasons:

\begin{itemize}
  \item The mean squared error (MSE) is almost always smaller than the BF or
    CL methods.
  \item Better approximation of the exact Bayesian procedure.
  \item Superior to the CL method since it give more weight to the a priori
    expectation of ultimate losses.
  \item Superior to the BF method since it gives more weight to actual loss
    experience.
\end{itemize}

\subsubsection{Mean Squared Error (MSE)}

The Benktander reserve $R_{GB}$ has a smaller mean squared error
(MSE) than $R_{BF}$ whenever $c^* > p_k / 2$ holds.

$c^*$ is the optimal credibility reserve credibility factor.

\newpage

\section{Hurlimann }

\begin{center}
  \begin{tabular}{|r|l|l|l|l|l|}
    \hline
    & \multicolumn{5}{c|}{development period} \\
    \hline
    period (i) & 1 & 2 & 3 & 4 & 5 \\
    \hline
    1 & S\_\{1,1\} & \cellcolor{blue!50}S\_\{1,2\} &
    \cellcolor{yellow!20}S\_\{1,3\} &
    \cellcolor{yellow!80}S\_\{1,4\} & \cellcolor{purple!20}S\_\{1,5\} \\
    \hline
    2 & \cellcolor{blue!50}S\_\{2,1\} & \cellcolor{yellow!20}S\_\{2,2\} &
    \cellcolor{yellow!80}S\_\{2,3\} & \cellcolor{purple!20}S\_\{2,4\} &  \\
    \hline
    3 & \cellcolor{yellow!20}S\_\{3,1\} & \cellcolor{yellow!80}S\_\{3,2\} &
    \cellcolor{purple!20}S\_\{3,3\} &  &  \\
    \hline
    4 & \cellcolor{yellow!80}S\_\{4,1\} &
    \cellcolor{purple!20}S\_\{4,2\} &  &  &  \\
    \hline
    5 & \cellcolor{purple!20}S\_\{5,1\} &  &  &  &  \\
    \hline
  \end{tabular}
\end{center}

Note that each of the highlighted diagonal rows represent the same
calendar development period.

\end{document}
