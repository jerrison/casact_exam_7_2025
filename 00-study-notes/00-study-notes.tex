\documentclass{article}
\usepackage{graphicx} % Required for inserting images
\usepackage{amsmath}
\usepackage[margin=1in]{geometry}
\usepackage{xcolor}
\usepackage{cancel}


\title{CASACT Exam 7 Notes}
\author{Jerrison Li }
\date{January 2025}

\begin{document}

\maketitle

\section{Benktander Method}

\subsection{General Relationship Between Reserve \& Ultimate Loss Estimates }

Suppose that $C_k$ is the actual claims amount paid after \textit{k} years of
development. Given a \textbf{reserve estimate} $\widehat{R}$ and 
\textbf{ultimate loss estimate} $\widehat{U}$, we have the following general
relationship:

\begin{equation}
\widehat{U} = C_k + \widehat{R}
\end{equation}

This relationship always holds. Note that $C_k$ is the cumulative paid amount.



\subsection{Bornhuetter/Ferguson (BF) Method }

The \textbf{Bornhuetter/Ferguson (BF) Method} estimates reserves based on an
\textbf{a priori} expectation of ultimate losses. Mathematically:

\begin{equation}
R_{BF} = q_k * U_0
\end{equation}

where:

\begin{itemize}
    \item $R_{BF}$ is the \textbf{reserve estimate}
    \item $q_k = 1 - \frac{1}{CDF}$ is the proportion of the ultimate claims
amount which is expected to remain unpaid after \textit{k} years of development
    \item $U_0$ is the a priori expectation of ultimate losses
\end{itemize}

Since $R_{BF}$ uses $U_0$, it assumes that the current claims amount $C_k$ is
\textbf{NOT predictive of future claims}.

\bigskip

Using the general relationship described above, we obtain the \textbf{BF
ultimate loss}:

\begin{equation}
U_{BF} = C_k + R_{BF}
\end{equation}

\subsection{Chain Ladder Method }

The \textbf{Chain Ladder Method} estimates ultimate losses and reserves based
on claims to date. In other words, it assumes that the current claims amount
$C_k$ is \textbf{fully predictive of future claims}. Mathematically:

\begin{equation}
U_{CL} = \frac{C_k}{p_k}
\end{equation}

\begin{equation}
R_{CL} = q_k * U_{CL}
\end{equation}

\subsection{Benktander Method }

Since the CL and BF methods represent extreme positions, where the CL method
fully believes $C_k$ and the BF method does not rely on $C_k$ at all, Gunnar
Benktander replaced $U_0$ with a credibility mixture:

\begin{equation}
U_c = c * U_{CL} + (1 - c) * U_0
\end{equation}

where $c$ is the credibility weight.

\bigskip

As the claims $C_k$ develop, credibility should increase, Benktander proposed 
the following:

\begin{itemize}
    \item Set $c = p_k$
    \item Set $R_{GB} = R_{BF} * \frac{U_{pk}}{U_0}$
\end{itemize}

\begin{equation}
    \begin{aligned}
    R_{GB} &= \color{blue}{R_{BF}} &* \color{black}\frac{U_{pk}}{U_0} \\
    &= (\color{blue}{q_k * \cancel{U_o}}) &* 
        \color{black}\frac{U_{pk}}{\cancel{U_o}} \\
    &= q_k * U_{pk}
    \end{aligned}
\end{equation}

\subsubsection{BF Method as a Credibility-Weighted Average}
    
Using our credibility mixture with $c = p_k$, we can show the following:

\begin{equation}
    \begin{aligned}
    U_{pk} &= p_k * U_{CL} + (1 - p_k) * U_0 \\
    &= p_k * U_{CL} + q_k * U_0 \\
    &= C_k + R_{BF} \\
    &= U_{BF}
    \end{aligned}
\end{equation}

then 

\begin{equation}
    \begin{aligned}
    R_{GB} &= q_k * U_{BF} \\
    \end{aligned}
\end{equation}

Hence, the \textbf{BF method} is a credibility-weighted average of the CL method
and the a priori expectation.

\subsubsection{Benktander Method as a Credibility-Weighted Average}

The \textbf{Benktander method} is a credibility-weighted average of the CL and
BF methods:

\begin{equation}
    \begin{aligned}
    U_{GB} &= C_k & + R_{GB} \\
    &= \color{blue}{C_k} & + q_k * U_{BF} \\
    &= \color{blue}{(1-q_k) * U_{CL}} & + q_k * U_{BF} \\
    % &= \color{blue}{p_k * U_{CL}} & + q_k * q_k * U_0 \\
    % &= \color{blue}{(1-q_k) * U_{CL}} & + q_k * q_k * U_0 \\
    % &= \color{blue}{(1-q_k) * U_{CL}} & + q_k^2 * U_0 \\
    \end{aligned}
\end{equation}

The \textbf{Benktander reserve} is also a credibility-weighted average of the CL
and BF methods:

\begin{equation}
    \begin{aligned}
    R_{GB} &= (1-q_k) * R_{CL} + q_k^2 * R_0 \\
    \end{aligned}
\end{equation}

We can also express the Benktander method as a credibility-weighted average of
the CL method and the a priori expectation:

\begin{equation}
    \begin{aligned}
    U_{GB} &= C_k  + R_{GB} \\
    &= \color{blue}{U_{CL} - \color{red}R_{CL}}  + q_k * U_{pk} \\
    &= \color{blue}{U_{CL} - \color{red}q_k * U_{CL}}  + q_k * ([1-q_k] * U_{CL} + q_k * U_0) \\
    &= \color{blue}{U_{CL} - \cancel{\color{red}q_k * U_{CL}}}  + \cancel{q_k * U_{CL}} - q_k^2 * U_{CL} + q_k^2 * U_0 \\
    &= \color{blue}{U_{CL}}   \color{black}- q_k^2 * U_{CL} + q_k^2 * U_0 \\
    &= (1-q_k^2) * U_{CL} + q_k^2 * U_0 \\
    &= U_{1-q_k^2}
    \end{aligned}            
\end{equation}

\end{document}
